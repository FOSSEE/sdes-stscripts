\documentclass{beamer}
\usepackage[latin1]{inputenc}
\usepackage[T1]{fontenc}
\usepackage{fixltx2e}
\usepackage{graphicx}
\usepackage{longtable}
\usepackage{float}
\usepackage{wrapfig}
\usepackage{soul}
\usepackage{textcomp}
\usepackage{marvosym}
\usepackage{wasysym}
\usepackage{latexsym}
\usepackage{amssymb}
\usepackage{hyperref}
\tolerance=1000
\usepackage[english]{babel} \usepackage{ae,aecompl}
\usepackage{mathpazo,courier,euler} \usepackage[scaled=.95]{helvet}
\usepackage{listings}
\lstset{
  language=TeX,
  basicstyle=\ttfamily\bfseries,
  commentstyle=\ttfamily\color{blue},
  stringstyle=\ttfamily\color{orange},
  showstringspaces=false,
  breaklines=true,
  postbreak = \space\dots
}


\mode<presentation>
{
  \usetheme{Warsaw}
  \useoutertheme{infolines}
  \setbeamercovered{transparent}
}

\newcommand{\inctime}[1]{\addtocounter{time}{#1}{\tiny \thetime\ m}}

\newcommand{\typ}[1]{\lstinline{#1}}

\newcommand{\kwrd}[1]{ \texttt{\textbf{\color{blue}{#1}}}  }
\title [{\LaTeX} for mathematics \& beyond] {{\LaTeX} for mathmatical formulae, bibliography and presentations}
\author {FOSSEE}
%%%%%%%%%%%%%%%%%%%%%%%%%%%%%%%%%%%%%%%%%%%%%%%%%%%%%%%%%%%%%%%%%%%%%%
% DOCUMENT STARTS
\begin{document}


\begin{frame}
  \begin{center}
    \vspace{12pt}
    \textcolor{blue}{\huge {\LaTeX} for mathmatical formulae, bibliography and presentations}
  \end{center}
  \vspace{18pt}
  \begin{center}
    \vspace{10pt}
    \includegraphics[scale=0.95]{../images/fossee-logo.png}\\
    \vspace{5pt}
    \scriptsize Developed by FOSSEE Team, IIT-Bombay. \\ 
    \scriptsize Funded by National Mission on Education through ICT\\
    \scriptsize  MHRD,Govt. of India\\
    \includegraphics[scale=0.30]{../images/iitb-logo.png}\\
  \end{center}
\end{frame}

\begin{frame}
  \frametitle{Objectives}
  \label{sec-2}
    At the end of this tutorial, you will be able to,
  \begin{itemize}
    \item Write simple mathematical formulae in {\LaTeX}.
    \item Typeset simple mathematical formulae in {\LaTeX}.
    \item Write bibliography for a LaTeX document.
    \item Make presentations in LaTeX, using beamer.
  \end{itemize}
\end{frame}

\begin{frame}[fragile]
  \frametitle{\LaTeX\ \&\ Mathematics : An Introduction}
  \begin{itemize}
  \item Math is enclosed in a pair of \lstinline{$} signs or %%$
    \lstinline+\(  \)+ 
  \item Used for typesetting inline Math. 
  \item \lstinline+\usepackage{amsmath}+
  \end{itemize}
\end{frame}

\begin{frame}[fragile]
  \frametitle{Matrices}
  \begin{itemize}
  \item \lstinline+\bmatrix+ is used to typeset the matrix A
  \item It works similar to the tabular environment
  \item \lstinline+&+ for demarcating columns
  \item \lstinline+\\+ for demarcating rows
  \item Other matrix environments
  \begin{table}
    \center
    \begin{tabular}{|l|l|}
      \hline
        Matrix Type & delimiters \\
      \hline
      \lstinline+matrix+  &  none\\
      \hline
      \lstinline+pmatrix+ &  \lstinline+(+\\
      \hline
      \lstinline+Bmatrix+ &  \lstinline+{+\\
      \hline
      \lstinline+vmatrix+ &  \lstinline+|+\\  
      \hline
      \lstinline+Vmatrix+ &  \lstinline+||+\\
      \hline
    \end{tabular}
  \end{table}
  \end{itemize}
\end{frame}

\begin{frame}[fragile]
  \frametitle{Matrices \ldots}
  \tiny{
  \begin{verbatim}
    \documentclass{article}
    \usepackage {amsmath}
    \begin{document}
      Welcome to the world of matrices\\ 
      $
      \begin{bmatrix}
        a & b \\
        c & d \\
      \end{bmatrix}
      \begin{vmatrix}
        e & f \\
        g & h
      \end{vmatrix}
     $
    \begin{equation}
      \begin{Vmatrix}
        i & j & k & l
      \end{Vmatrix}
    \end{equation}
    \begin{equation}
      \begin{pmatrix}
        m & n & o & p
      \end{pmatrix}
    \end{equation}
    \end{document}
  \end{verbatim}
  }
\end{frame}

\begin{frame}[fragile]
  \frametitle{Superscripts \& Subscripts}
  \begin{itemize}
  \item \lstinline+^+ for superscripts
  \item \lstinline+_+ for subscripts
  \item Enclose multiple characters in \lstinline+{ }+
  \end{itemize}
  \begin{verbatim}
    \documentclass{article}
    \begin{document}
      $ i^j $ $k_l$ \\
      $ {{{a^b}^c}^d} \\ {z_{y_{x_w}}} $ \\
    \end{document}
  \end{verbatim}
\end{frame}

\begin{frame}[fragile]
  \frametitle{Summation \& integration}
  \begin{itemize}
  \item \lstinline+\sum+ command gives the summation symbol
  \item The upper and lower limits are specified using the
    \lstinline+^+ and \lstinline+_+ symbols. 
  \item Similarly the integral symbol is obtained using
    \lstinline+\int+ command. 
  \end{itemize}
  \tiny{
  \begin{verbatim}
    \documentclass{article}
    \usepackage {amsmath}
    \begin{document}
      We note that \\
      (a) Summations in text style \\
      $ \sum_{1}^{10} a $ \\
      (b) Summations in display style \\
      \begin{equation}
        \sum_{1}^{10} a
      \end{equation}
      look a lot different when rendered.\\
      A similar analogy holds true for integrals, when using \\
      $ \int_{1}^{10} a $ \\
      and \\
      \begin{equation}
        \int_{1}^{10} a
      \end{equation}
    \end{document}
  \end{verbatim}
  }
\end{frame}

\begin{frame}[fragile]
  \frametitle{\lstinline+displayed+ math}
  \begin{itemize}
  \item Display equations are the other type of displaying math
  \item \LaTeX~ or \lstinline+amsmath+ has a number of environments
    for ``displaying'' equations, with minor differences. 
  \item In general, enclose math in \lstinline+\[+ and \lstinline+\]+
    to get displayed math. 
  \item \lstinline+\begin{equation*}+ is equivalent to this.
  \item Use \lstinline+\begin{equation}+ to get numbered
    equations. %%\end{equation} 
  \end{itemize}
\end{frame}

\begin{frame}[fragile]
  \frametitle{Groups of equations}
  \begin{itemize}
  \item The \lstinline+equation+ environment allows typesetting of
    just 1 equation. 
  \item \lstinline+eqnarray+ allows typesetting of multiple equations 
  \item It is similar to the \lstinline+table+ environment
  \item The parts of the equation that need to be aligned are
    indicated using \& symbol.
  \item Each equation is separated by a \lstinline+\newline+ command
  \end{itemize}
  \tiny{
  \begin{verbatim}
    \documentclass{article}
    \begin{document}
      \begin{eqnarray}
        a & = & b + c +d\\
        & = & d + e
      \end{eqnarray}
    \end{document}
  \end{verbatim}
  }
\end{frame}

\begin{frame}[fragile]
  \frametitle{Fractions \& Surds}
  \begin{itemize}
  \item Fractions are typeset using \lstinline+\frac+ command 
  \item \lstinline+\frac{numerator}{denominator}+ is typeset as
    $\frac{numerator}{denominator}$
  \item Surds are typeset using \lstinline+\sqrt[n]+ command
  \end{itemize}
  \tiny{
    \begin{verbatim}
      \documentclass{article}
      \begin{document}
        The glass is $\frac{3}{4}$ full.\\
        \begin{equation}
          \frac{3}{4}
        \end{equation}
        We now move on to square root and cube root of 2
        \begin{eqnarray}
          \sqrt{2}\\
          \sqrt[3]{2}
        \end{eqnarray}
      \end{document}
    \end{verbatim}
  }
\end{frame}

\begin{frame}[fragile]
  \frametitle{Greek characters \& Spacing}
  \begin{itemize}
  \item Typesetting Greek characters is simple
  \item \lstinline+\alpha+, \lstinline+\beta+, \lstinline+\gamma+,
    \ldots \lstinline+\Alpha+, \lstinline+\Beta+, \lstinline+\Gamma+
    \ldots 
  \item To get additional spacing in Math environments ---
    \begin{center}
      \begin{tabular}{|l|l|l|}
        \hline
        Abbrev. & Spelled out & Example  \\
        \hline
        \lstinline+\,+ & \lstinline+\thinspace+ & $A\,B$ \\
        \hline
        \lstinline+\:+ & \lstinline+\medspace+ & $A\:B$ \\
        \hline
        \lstinline+\;+ & \lstinline+\thickspace+ & $A\;B$ \\
        \hline
        & \lstinline+\quad+ & $A \quad B$ \\
        \hline
        & \lstinline+\qquad+ & $A \qquad B$ \\
        \hline
        \lstinline+\!+ & \lstinline+\negthinspace+ & $A!B$ \\
        \hline
        & \lstinline+\negmedspace+ & $A \negmedspace B$ \\
        \hline
        & \lstinline+\negthickspace+ & $A \negthickspace B$ \\
        \hline
      \end{tabular}
    \end{center}
  \end{itemize}
\end{frame}

\section{Bibliography}
\begin{frame}[fragile]
  \frametitle{Bibliography}
  \begin{itemize}
  \scriptsize{
  \item \lstinline+thebibliography+ environment provides a clean and
    simple way to add a bibliography to \LaTeX documents. 
  \item \lstinline+\begin{thebibliography}+ takes as argument the
    maximum width of the label that references will have. 
  \item Each item of the Bibliography is similar to an item in a
    list. 
  \item \lstinline+\bibitem[label]{name}+ followed by the actual
    reference info. 
  \item label replaces auto enumeration numbers 
  \item \lstinline+\cite{name}+ is used to \lstinline+cite+ the
    \lstinline+bibitem+ 
  \item You will need to compile twice. 
  }
  \end{itemize}
  \tiny{
    \begin{verbatim}
      \documentclass{article}
      \begin{document}
        Official Sources~\cite{Official}
        indicate that \ldots
        \begin{thebibliography}{99}
          \bibitem{Official} H.~Partl:
            \emph{Dept of Defence Notfication},
             Volume~9, Issue~1 (1988)
        \end{thebibliography}
      \end{document}
  \end{verbatim}
  }
\end{frame}

\section{Presentations - Beamer}
\begin{frame}[fragile]
  \frametitle{Beamer}
  \begin{itemize}
  \item Use beamer since your report's \LaTeX~ would be re-usable.
  \item It is recommended to start with one of the beamer templates.
  \item \lstinline+\documentclass{beamer}+ tells \LaTeX~ to start a
    beamer presentation. 
  \item A beamer document is very similar to any other \LaTeX~
    document except that content is divided into slides. 
  \end{itemize}
\end{frame}

\begin{frame}[fragile]
  \frametitle{Beamer \ldots}
  \begin{itemize}
  \item \lstinline+\usetheme+ command is used to specify the theme of the
    presentation. 
  \item \lstinline+\usecolortheme+ command is used to specify the color
    theme. 
  \item The content of a slide is enclosed within
    \lstinline+\begin{frame}{Title}{Subtitle}+ and
    \lstinline+\end{frame}+ 
  \item If the slide contains \lstinline+verbatim+
    \lstinline+lstlisting+ environments, the \lstinline+\begin{frame}+
    should be passed an additional argument \lstinline+[fragile]+
  \item Overlays can be achieved using the \lstinline+\pause+
    command. 
  \item To achieve more with beamer, it is highly recommended that you
    look at the \texttt{beameruserguide} 
  \end{itemize}
\end{frame}

\begin{frame}[fragile]
  \frametitle{Beamer \ldots}
  \tiny{
    \begin{verbatim}
      \documentclass{beamer}
      \usetheme{Warsaw}
      \useoutertheme{infolines}
      \title{Make a LaTeX presentation using Beamer}
      \author{FOSSEE}
      \institute{IIT Bombay}
      \date{}
      \begin{document}
        \begin{frame}
          \titlepage
        \end{frame}
        \begin{frame}{Introduction}
          This is a short introduction to Beamer class.
        \end{frame}
        \begin{frame}[fragile]
          \frametitle{Introduction to \LaTeX}
            Most non trivial \LaTeX documents begins with the following or similar lines
          \begin{itemize}
            \pause \item \backslash documentclass{foo}
             \pause \item \backslash \usepackage{bar}
          \end{itemize}
        \end{frame}
      \end{document}
    \end{verbatim}
  }
\end{frame}

\begin{frame}[fragile]
  \frametitle{\typ{Summary}}
  \label{sec-8}
  In this tutorial, we have,
  \begin{itemize}
    \item Written simple mathematical formulae in {\LaTeX}.
    \item Typeset simple mathematical formulae in {\LaTeX}.
    \item Written bibliography for a LaTeX document.
    \item Made a sample presentations in LaTeX, using beamer.
  \end{itemize}
\end{frame}

\begin{frame}[fragile]
  \frametitle{\typ{Evaluation}}
  \label{sec-9}
  {\small 
  \begin{enumerate}
    \item What is the function of useoutertheme command used in the beamer 
presentation example shown before ?? what happens when you comment out or
remove the line. 
    \item Are commands like $ \backslash alpha, \backslash beta, etc $ commands provided by amsmath package ? 
  \end{enumerate}
  }
\end{frame}

\begin{frame}[fragile]
\frametitle{\typ{Solutions}}
\label{sec-10}
\begin{enumerate}
  \item The outertheme command in beamer is used to customize the amount of 
header/footer information shown in each slide. In the example shown below the
useoutertheme command with infolines argument automatically adds more 
information to the footer like page number, author and institute,etc.
  \item No, commands like alpha, beta, etc are not commands provided by the amsmath
package.
\end{enumerate}
\end{frame}

\begin{frame}
  \begin{block}{}
  \begin{center}
    \textcolor{blue}{\Large THANK YOU!} 
  \end{center}
  \end{block}
  \begin{block}{}
    \begin{center}
      For more Information, visit our website\\
      \url{http://fossee.in/}
    \end{center}  
  \end{block}
\end{frame}

\end{document}


