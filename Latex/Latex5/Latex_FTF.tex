%%%%%%%%%%%%%%%%%%%%%%%%%%%%%%%%%%%%%%%%%%%%%%%%%%%%%%%%%%%%%%%%%%%%%%%%%%%%%%%%
% LateX
%
% Author: FOSSEE 
% Copyright (c) 2009, FOSSEE, IIT Bombay
%%%%%%%%%%%%%%%%%%%%%%%%%%%%%%%%%%%%%%%%%%%%%%%%%%%%%%%%%%%%%%%%%%%%%%%%%%%%%%%%
\documentclass[17pt,compress]{beamer}
\usepackage{beamerthemesplit}
\usepackage[latin1]{inputenc}
\usepackage[T1]{fontenc}
\usepackage{fixltx2e}
\usepackage{graphicx}
\usepackage{longtable}
\usepackage{float}
\usepackage{wrapfig}
\usepackage{soul}
\usepackage{textcomp}
\usepackage{marvosym}
\usepackage{wasysym}
\usepackage{latexsym}
\usepackage{amssymb}
\usepackage{hyperref}
\tolerance=1000
\usepackage[english]{babel} \usepackage{ae,aecompl}
\usepackage{mathpazo,courier,euler} \usepackage[scaled=.95]{helvet}
\usepackage{listings}

\lstset{
  language=TeX,
  basicstyle=\ttfamily\bfseries,
  commentstyle=\ttfamily\color{blue},
  stringstyle=\ttfamily\color{orange},
  showstringspaces=false,
  breaklines=true,
  postbreak = \space\dots
}

\mode<presentation>
{
  \usetheme{Warsaw}
  \useoutertheme{infolines}
  \setbeamercovered{transparent}
  \setbeamertemplate{navigation symbols}{}
}
% Taken from Fernando's slides.
\usepackage{ae,aecompl}
\usepackage[scaled=.95]{helvet}

\usepackage[english]{babel}
\usepackage[latin1]{inputenc}
\usepackage[T1]{fontenc}

% change the alerted colour to LimeGreen
\definecolor{LimeGreen}{RGB}{50,205,50}
\setbeamercolor{structure}{fg=LimeGreen}
\author[FOSSEE]{}
\institute[IIT Bombay]{}
\date[]{}
% \setbeamercovered{transparent}

% theme split
\usepackage{verbatim}
\newenvironment{colorverbatim}[1][]%
{%
\color{blue}
\verbatim
}%
{%
\endverbatim
}%

\usepackage{mathpazo,courier,euler}
\usepackage{listings}
\lstset{language=sh,
    basicstyle=\ttfamily\bfseries,
  showstringspaces=false,
  keywordstyle=\color{black}\bfseries}

% logo
\logo{\includegraphics[height=1.30 cm]{../images/3t-logo.pdf}}
\logo{\includegraphics[height=1.30 cm]{../images/fossee-logo.pdf}

\hspace{7.5cm}
\includegraphics[scale=0.99]{../images/fossee-logo.pdf}\\
\hspace{281pt}
\includegraphics[scale=0.80]{../images/3t-logo.pdf}}
\newcommand{\typ}[1]{\lstinline{#1}}


\begin{document}

\sffamily \bfseries
\title
[{\LaTeX}: Tables \& Figures]
{{\LaTeX}: Tables \& Figures}
\author
[FOSSEE]
{\small Talk to a Teacher\\{\color{blue}\url{http://spoken-tutorial.org}}
\\\vspace{0.25cm}National Mission on Education
 through ICT\\{\color{blue}\url{ http://sakshat.ac.in}} \\ [1.65cm]
   Contributed by FOSSEE Team \\IIT Bombay  \\[0.3cm]
}

\begin{frame}
   \titlepage
\end{frame}


\begin{frame}
  \frametitle{Objectives}
  In this tutorial we will learn how to:
  \begin{itemize}
  \item Add graphics in Latex document
  \item Include tabular environments in Latex document
  \end{itemize}
\end{frame}


\begin{frame}
  \frametitle{Prerequisites}
  Spoken Tutorial on:
  \begin{enumerate}
  \item {\LaTeX} - Installation
  \item {\LaTeX} - Introduction
  \item {\LaTeX} - Basics \& Structuring
  \item {\LaTeX} - Typesetting Text
  \end{enumerate}
\end{frame}

\begin{frame}[fragile]
  \frametitle{Figures}
  \begin{itemize}
  \item The \typ{graphicx} package allows us to insert graphics
  \item \lstinline+\usepackage{graphicx}+
  \item To add a graphic, use \lstinline{\includegraphics} command
  \item Use relative path to the image
  \end{itemize}
\end{frame}

\begin{frame}[fragile]
  \frametitle{\lstinline{includegraphics}}
  It takes following optional arguments
  \begin{itemize}
  \item \lstinline+height+, \lstinline+width+ -- If only one of them
    is specified, aspect ratio is maintained 
  \item \lstinline+keepaspectratio+ -- boolean value to keep aspect
    ratio or not 
  \item \lstinline+angle+ -- specify by what angle the image should
    be rotated 
  \end{itemize}
\end{frame}

\begin{frame}[fragile]
  \frametitle{\lstinline{includegraphics..}}
    Syntax for \lstinline{includegraphics} command with optional arguments and 
    relative path to the image:
  \begin{itemize}
  \item \lstinline{\includegraphics[<args>]}\{<img path> \}
  \end{itemize}
\end{frame}

\begin{frame}[fragile]
  \frametitle{Floats}
  \begin{itemize}
  \item Graphics (\& Tables) are special
  \item They are ``floated'' to the next page
  \item Enclose graphic within \lstinline+figure+ environment to make
    it float 
  \item Figure environment takes additional parameter for location of
    float 
  \end{itemize}
\end{frame}

\begin{frame}[fragile]
  \frametitle{Floats..}
\begin{table}
    \begin{tabular}{|c|l}
      Specifier & Permission\\\hline
      t & Top of page\\
      b & Bottom of page\\
      p & Separate page for floats\\
      h & Here (the same place)\\
      ! & Overrides internal parameters
    \end{tabular}
  \end{table}
\end{frame}


\begin{frame}
  \frametitle{Captions and References}
  \begin{itemize}
  \item \lstinline+caption+ to add captions to figures
  \item To place the image in the center, we enclose it in the
    \lstinline+center+ environment 
  \item We can label images too
  \item Add label after caption command
  \item Figures are auto numbered
  \end{itemize}
\end{frame}

\begin{frame}[fragile]
  \frametitle{Captions and References..}
  Sample code to include caption and label in a center aligned figure:
  \vspace{8pt}
  {\tiny
    \begin{verbatim}
      \begin{figure}
          \begin{center}
            \includegraphics{some image}
          \end{center}
          \caption{A caption for some image !}
          \label{Label for some image !}
      \end{figure}
    \end{verbatim}
  }
\end{frame}  


\begin{frame}[frame]
  \frametitle{Tables}
  \begin{itemize}
  \item \lstinline+tabular+ is used to typeset a table
  \item It is enclosed in a \lstinline+table+ environment to make it a
    float 
  \item \lstinline+table+ environment also gives captions, labels \& auto
    numbering  
  \end{itemize}
\end{frame}


\begin{frame}[fragile]
  \frametitle{\lstinline+tabular+}
  \begin{itemize}
  \item Tabular takes formatting of each column as argument
  \end{itemize}
  \begin{table}
    \begin{tabular}{|l|l|}
      \lstinline+l+ & left justified column content\\\hline
      \lstinline+r+ & right justified column content\\\hline
      \lstinline+c+ & centered column content\\\hline
      \lstinline+|+ & produces a vertical line\\
    \end{tabular}
  \end{table}
\end{frame}

\begin{frame}[fragile]
  \frametitle{\lstinline+tabular+..}
\begin{itemize}
  \item Also takes an optional parameter for specifying position of
    table 
  \item \lstinline+t+ for top, \lstinline+b+ for bottom, \lstinline+c+
    for center 
  \item Seperate each column of a table by '\&'
  \item Each row is separated by newline \lstinline{\\}
  \item \lstinline+\hline+ give a horizontal line between two rows
  \end{itemize}
\end{frame}


\begin{frame}[fragile]
  \frametitle{\lstinline+tabular+..}
  A sample code that shows complete use of tabular environment:
  \vspace{8pt}
  {\tiny
    \begin{verbatim}
      \begin{center}
          \begin{tabular}{ l | c || r | }
            \hline
            1 & 2 & 3 \\ \hline
            4 & 5 & 6 \\ \hline
            7 & 8 & 9 \\
            \hline
          \end{tabular}
      \end{center}
    \end{verbatim}
  }
\end{frame}  

\begin{frame}[fragile]
  \frametitle{List of Tables, Figures}
  \begin{itemize}
  \item \lstinline+\listoftables+ -- to add a list of tables
  \item \lstinline+\listoffigures+ -- to add a list of figures
  \end{itemize}
\end{frame}

\begin{frame}[fragile]
	\frametitle{Summary}
    We learned how to,
	\begin{itemize}
	\item Add graphics to a LateX document
	\item Include tabular environments in a LateX document
	\end{itemize}
\end{frame}

\begin{frame}[fragile]
\frametitle{Evaluation}
\begin{enumerate}
\item Which input parameter is used in the figure environment to make it float
      to the bottom of the page ?
\item What is the mandatory argument in tabular environment specification ?
\end{enumerate}
\end{frame}
\begin{frame}

\frametitle{Solutions}
\begin{enumerate}
\vspace{15pt}
\item Input parameter `b' is passed as argument, to make it float to the bottom
      of the page.
\item It is mandatory to specify alignment of each column in tabular 
      environment.
\end{enumerate}
\end{frame}

\begin{frame}
\frametitle{SDES \& FOSSEE}
\begin{center}
\begin{itemize}
\item \small{SDES}\\
\small{\color{LimeGreen}Software Development techniques for Engineers and Scientists} \\
\scriptsize An initiative by FOSSEE. \\
\vspace{3pt}
\scriptsize For more information on SDES, please visit {\color{blue}\url{http://fossee.in/sdes}}\\
\vspace{12pt}
\item \small{FOSSEE}\\
\small {\color{LimeGreen}Free and Open-source Software for \\Science and Engineering Education} \\
\scriptsize Based at IIT Bombay, Funded by MHRD.\\
\vspace{3pt}
\scriptsize Part of National Mission on Education through ICT (NME-ICT) \\
\end{itemize}
\end{center}
\end{frame}

\begin{frame}
\frametitle{About the Spoken Tutorial Project}
\begin{itemize}
\item Watch the video available at {\color{blue}\url{http://spoken-tutorial.org /What\_is\_a\_Spoken\_Tutorial}} 
\item It summarises the Spoken Tutorial project 
\item If you do not have good bandwidth, you can download and watch it
\end{itemize}
\end{frame}

\begin{frame}
\frametitle{Spoken Tutorial Workshops}The Spoken Tutorial Project Team 
\begin{itemize}
\item Conducts workshops using spoken tutorials 
\item Gives certificates to those who pass an online test 
\item For more details, please write to \\ \hspace {0.5cm}{\color{blue}contact@spoken-tutorial.org}
\end{itemize}
\end{frame}

\begin{frame}
\frametitle{Acknowledgements}
\begin{itemize}
\item Spoken Tutorial Project is a part of the Talk to a Teacher  project 
\item It is supported by the National Mission on Education through  ICT, MHRD, Government of India 
\item More information on this Mission is available at: \\{\color{blue}\url{http://spoken-tutorial.org/NMEICT-Intro}}
\end{itemize}
\end{frame}

\begin{frame}
  \begin{block}{}
  \begin{center}
  {\Large THANK YOU!} 
  \end{center}
  \end{block}
\begin{block}{}
  \begin{center}
    For more Information, visit our website\\
    {\color{blue}\url{http://fossee.in/}}
  \end{center}  
  \end{block}
\end{frame}

\end{document}
