%%%%%%%%%%%%%%%%%%%%%%%%%%%%%%%%%%%%%%%%%%%%%%%%%%%%%%%%%%%%%%%%%%%%%%%%%%%%%%%%
% LateX
%
% Author: FOSSEE 
% Copyright (c) 2009, FOSSEE, IIT Bombay
%%%%%%%%%%%%%%%%%%%%%%%%%%%%%%%%%%%%%%%%%%%%%%%%%%%%%%%%%%%%%%%%%%%%%%%%%%%%%%%%

\documentclass[14pt,compress]{beamer}

\mode<presentation>
{
  \usetheme{Warsaw}
  \useoutertheme{infolines}
  \setbeamercovered{transparent}
}

\usepackage[english]{babel}
\usepackage[latin1]{inputenc}
%\usepackage{times}
\usepackage[T1]{fontenc}

% Taken from Fernando's slides.
\usepackage{ae,aecompl}
\usepackage{mathpazo,courier,euler}
\usepackage[scaled=.95]{helvet}

\definecolor{darkgreen}{rgb}{0,0.5,0}

\usepackage{listings}
\lstset{language=bash,
    basicstyle=\ttfamily\bfseries,
    commentstyle=\color{red}\itshape,
  stringstyle=\color{darkgreen},
  showstringspaces=false,
  keywordstyle=\color{blue}\bfseries}

\newcommand{\inctime}[1]{\addtocounter{time}{#1}{\tiny \thetime\ m}}

\newcommand{\typ}[1]{\lstinline{#1}}

\newcommand{\kwrd}[1]{ \texttt{\textbf{\color{blue}{#1}}}  }

\setbeamercolor{emphbar}{bg=blue!20, fg=black}
\newcommand{\emphbar}[1]


\begin{document}

\begin{frame}

\begin{center}
\vspace{12pt}
\textcolor{blue}{\huge {\LaTeX}: Installation}
\end{center}
\vspace{18pt}
\begin{center}
\vspace{10pt}
\includegraphics[scale=0.95]{fossee-logo.png}\\
\vspace{5pt}
\scriptsize Developed by FOSSEE Team, IIT-Bombay. \\ 
\scriptsize Funded by National Mission on Education through ICT\\
\scriptsize  MHRD,Govt. of India\\
\includegraphics[scale=0.15]{images/iitb-logo.jpg}\\
\end{center}
\end{frame}

\begin{frame}
  \frametitle{Objectives}
  At the end of this session, you will
  \begin{itemize}
  \item Learn how to install LaTeX
  \item Learn how to install a TeX editor plug-in
  \item Know how to configure the TeX editor for LaTeX
  \item Know some useful information on LaTeX packages.
  \item Be able to complie a TeX file to pdf.

  \end{itemize}
\end{frame}

\begin{frame}
  \frametitle{Prerequisites}
  Prerequisites for Installation of LaTeX
  \begin{itemize}
  \item You'll need a TeX distribution.
  \item A good text editor and a DVI or PDF viewer.
  \item Updated Linux distribution as Operating System.
  \item Working Internet connection (recommended, for installation over the network).
  \end{itemize}
\end{frame}

\begin{frame}[fragile]
  \frametitle{TeX Distribution}
  TeX Live
  \begin{itemize}
  \item Easy way to get up and running with the TeX document production system.
  \end{itemize}
  Acquire TeX Live in many ways
  \begin{itemize}
  \item Getting TeX Live on DVD (recommended, for installation from media).
  \item Installing TeX Live over the Internet (recommended, for installation over the network).
  \end{itemize}
\end{frame}

\begin{frame}[fragile]
  \frametitle{Installation}
  Running the installer
  \begin{itemize}
  \item Run the \emph{install-tl} script to install.
  \end{itemize}
  Commands
  \begin{itemize}
  \item \emph{sudo apt-get install texlive} : Basic subset of TeX Live's functionality.
  \item \emph{sudo apt-get install texlive-full} : Complete TeX Live distribution.
  \end{itemize}
\end{frame}

\begin{frame}[fragile]
  \frametitle{Installation}
  Using a Package Manager
  \begin{itemize}
  \item Open \emph{Package Manager}
  \item Search for \emph{texlive-full}
  \item Mark for installation and apply
  \end{itemize}
\end{frame}

\begin{frame}[fragile]
  \frametitle{LaTeX Plug-in}
  Gedit has a plug-in for LaTeX which converts Gedit into a LaTeX editor
  \begin{itemize}
  \item \emph{sudo apt-get install gedit-latex-plugin} : Installs plug-in
  \item \emph Activate the plug-in : Click (Edit > Preferences > Plugins > Check LaTeX Plugin)
  \end{itemize}
\end{frame}

\begin{frame}[fragile]
  \frametitle{LaTeX Packages}
  Recommended LaTeX Packages
  \begin{itemize}
  \item \emph{latex-beamer} : Beamer package is used to create presentations.
  \item \emph{TeXPower} : Is a bundle of style and class files for creating dynamic online presentations with LaTeX.
  \item \emph{Prosper} : A LaTeX class for writing transparencies.
  \item \emph{texlive-pictures} : This is a LaTeX package for drawing graphics.
  \item \emph{texlive-latex-extra} : This is a large collection of add-on packages for LaTeX.
  \end{itemize}
\end{frame}

\begin{frame}[fragile]
  \frametitle{Compilation}
  LaTeX distribution installed, you may compile a LaTeX document
  \begin{itemize}
  \item \emph{pdflatex filename.tex} : Compiles the LaTeX document.
  \item Output : PDF file.
  \end{itemize} 
\end{frame}


\begin{frame}[fragile]
	\frametitle{Summary}
	\begin{itemize}
        \item Installing a LaTeX distribution.
        \item Install/Configure LaTeX Plug-in.
        \item Recommended LaTeX Packages.
	\item Compile a LaTeX document.
	\end{itemize}
\end{frame}

\begin{frame}
\begin{block}{}
  \begin{center}
  \textcolor{blue}{\Large THANK YOU!} 
  \end{center}
  \end{block}
\begin{block}{}
  \begin{center}
    For more Information, visit our website\\
    \url{http://fossee.in/}
  \end{center}  
  \end{block}
\end{frame}

\end{document}
