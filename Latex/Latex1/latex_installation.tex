\documentclass[17pt,compress]{beamer}
\usepackage{beamerthemesplit}
\mode<presentation>
{
  \usetheme{Warsaw}
  \useoutertheme{infolines}
  \setbeamercovered{transparent}
  \setbeamertemplate{navigation symbols}{}
}
% Taken from Fernando's slides.
\usepackage{ae,aecompl}
\usepackage[scaled=.95]{helvet}

\usepackage[english]{babel}
\usepackage[latin1]{inputenc}
\usepackage[T1]{fontenc}

% change the alerted colour to LimeGreen
\definecolor{LimeGreen}{RGB}{50,205,50}
\setbeamercolor{structure}{fg=LimeGreen}
\author[FOSSEE]{}
\institute[IIT Bombay]{}
\date[]{}
% \setbeamercovered{transparent}

% theme split
\usepackage{verbatim}
\newenvironment{colorverbatim}[1][]%
{%
\color{blue}
\verbatim
}%
{%
\endverbatim
}%

\usepackage{mathpazo,courier,euler}
\usepackage{listings}
\lstset{language=sh,
    basicstyle=\ttfamily\bfseries,
  showstringspaces=false,
  keywordstyle=\color{black}\bfseries}

% logo
\logo{\includegraphics[height=1.30 cm]{../images/3t-logo.pdf}}
\logo{\includegraphics[height=1.30 cm]{../images/fossee-logo.pdf}

\hspace{7.5cm}
\includegraphics[scale=0.99]{../images/fossee-logo.pdf}\\
\hspace{281pt}
\includegraphics[scale=0.80]{../images/3t-logo.pdf}}


\begin{document}

\sffamily \bfseries
\title
[\LaTeX: \: Installation]
{\LaTeX: \:Installation}
\author
[FOSSEE]
{\small Talk to a Teacher\\{\color{blue}\url{http://spoken-tutorial.org}}\\\vspace{0.25cm}National Mission on Education
 through ICT\\{\color{blue}\url{ http://sakshat.ac.in}} \\ [1.65cm]
   Contributed by FOSSEE Team \\IIT Bombay  \\[0.3cm]
}

% slide 1
\begin{frame}
   \titlepage
\end{frame}

\begin{frame}
  \frametitle{Objectives}
  At the end of this tutorial you will,
  \begin{itemize}
  \item Learn to install LaTeX
  \item Learn to install a TeX editor plug-in
  \item Learn to configure the TeX editor for LaTeX
  \item Know few LaTeX packages
  \item Be able to compile a TeX file to pdf

  \end{itemize}
\end{frame}

\begin{frame}
  \frametitle{Prerequisites}
  Prerequisites for Installation of LaTeX
  \begin{itemize}
  \item You'll need a TeX distribution
  \item Good text editor, DVI or PDF viewer
  \item Updated Linux distribution as OS
  \item Working Internet connection (for installation over the network)
  \end{itemize}
\end{frame}

\begin{frame}[fragile]
  \frametitle{TeX Distribution}
  TeX Live,
  \begin{itemize}
  \item Easy way to get up and running with the TeX document production system
  \item Acquire TeX Live in many ways
  \item Getting TeX Live on DVD (for installation from media)
  \item Installing TeX Live over the network
  \end{itemize}
\end{frame}

\begin{frame}[fragile]
  \frametitle{Installation}
  Running the installer
  \begin{itemize}
  \item Run the \emph{install-tl} script to install
  Commands
  \item \emph{sudo apt-get install texlive}: 
             Basic subset of TeX Live's functionality
  \item \emph{sudo apt-get install texlive-full}: 
             Complete TeX Live distribution
  \end{itemize}
\end{frame}

\begin{frame}[fragile]
  \frametitle{Installation..}
  Using a Package Manager
  \begin{itemize}
  \item Open \emph{Package Manager}
  \item Search for \emph{texlive-full}
  \item Mark for installation and apply
  \end{itemize}
\end{frame}

\begin{frame}[fragile]
  \frametitle{LaTeX Plug-in}
  Gedit has a plug-in for LaTeX which converts Gedit into a LaTeX editor
  \begin{itemize}
  \item \emph{sudo apt-get install gedit-latex-plugin} : Installs plug-in
  \item \emph Activate the plug-in: 
              Click (Edit > Preferences > Plugins > Check LaTeX Plugin)
  \end{itemize}
\end{frame}

\begin{frame}[fragile]
  \frametitle{LaTeX Packages}
  Recommended LaTeX Packages
  \begin{itemize}
  \item \emph{latex-beamer} : to create presentations
  \item \emph{TeXPower} : for dynamic online presentations
  \item \emph{Prosper} : class for writing transparencies
  \end{itemize}
\end{frame}

\begin{frame}[fragile]
  \frametitle{LaTeX Packages..}
  \begin{itemize}
  \item \emph{texlive-pictures} : for drawing graphics
  \item \emph{texlive-latex-extra} : add-on packages
  \end{itemize}
\end{frame}

\begin{frame}[fragile]
  \frametitle{Compilation}
  LaTeX distribution installed, you may compile a LaTeX document to pdf file
  \begin{itemize}
  \item \emph{pdflatex filename.tex} : Compiles the LaTeX document
  \item Output : PDF file
  \end{itemize} 
\end{frame}


\begin{frame}[fragile]
	\frametitle{Summary}
	\begin{itemize}
        \item Install LaTeX on your computer
        \item Install/Configure TeX editor with LaTeX plug-in
        \item Choose a LaTeX package as per requirement
	\item Compile a TeX file to PDF
	\end{itemize}
\end{frame}

\begin{frame}[fragile]
	\frametitle{Exercise}
	\begin{itemize}
        \item How can we check the version of the LaTeX package installed ?
        \item How can we check if the plug-in is properly configured with TeX 
              editor ?
	\end{itemize}
\end{frame}

\begin{frame}[fragile]
	\frametitle{Solutions}
	\begin{itemize}
        \item Use the command \textit {latex -v} to check the version installed
        \item Edit > Preferences > Plugins > Check LaTeX Plug-in
	\end{itemize}
\end{frame}

\begin{frame}
\frametitle{SDES \& FOSSEE}
\begin{center}
\begin{itemize}
\item \small{SDES}\\
\small{\color{LimeGreen}Software Development techniques 
    for Engineers and Scientists} \\
\scriptsize An initiative by FOSSEE. \\
\vspace{3pt}
\scriptsize For more information on SDES, please visit 
{\color{blue}\url{http://fossee.in/sdes}}\\
\vspace{12pt}
\item \small{FOSSEE}\\
\small {\color{LimeGreen}Free and Open-source Software for \\
Science and Engineering Education} \\
\scriptsize Based at IIT Bombay, Funded by MHRD.\\
\vspace{3pt}
\scriptsize Part of National Mission on Education through ICT (NME-ICT) \\
\end{itemize}
\end{center}
\end{frame}

\begin{frame}
\frametitle{About the Spoken Tutorial Project}
\begin{itemize}
\item Watch the video available at 
{\color{blue}\url{http://spoken-tutorial.org /What\_is\_a\_Spoken\_Tutorial}} 
\item It summarises the Spoken Tutorial project 
\item If you do not have good bandwidth, you can download and watch it
\end{itemize}
\end{frame}

\begin{frame}
\frametitle{Spoken Tutorial Workshops}The Spoken Tutorial Project Team 
\begin{itemize}
\item Conducts workshops using spoken tutorials 
\item Gives certificates to those who pass an online test 
\item For more details, please write to \\ \hspace {0.5cm}
{\color{blue}contact@spoken-tutorial.org}
\end{itemize}
\end{frame}

\begin{frame}
\frametitle{Acknowledgements}
\begin{itemize}
\item Spoken Tutorial Project is a part of the Talk to a Teacher  project 
\item It is supported by the National Mission on Education through 
ICT, MHRD, Government of India 
\item More information on this Mission is available at: \\
{\color{blue}\url{http://spoken-tutorial.org/NMEICT-Intro}}
\end{itemize}
\end{frame}

\begin{frame}
  \begin{block}{}
  \begin{center}
  {\Large THANK YOU!} 
  \end{center}
  \end{block}
\begin{block}{}
  \begin{center}
    For more Information, visit our website\\
    {\color{blue}\url{http://fossee.in/}}
  \end{center}  
  \end{block}
\end{frame}

\end{document}
