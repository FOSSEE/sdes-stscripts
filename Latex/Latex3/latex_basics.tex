\documentclass[17pt,compress]{beamer}
\usepackage{beamerthemesplit}
\mode<presentation>
{
  \usetheme{Warsaw}
  \useoutertheme{infolines}
  \setbeamercovered{transparent}
  \setbeamertemplate{navigation symbols}{}
}
% Taken from Fernando's slides.
\usepackage{ae,aecompl}
\usepackage[scaled=.95]{helvet}

\usepackage[english]{babel}
%\usepackage[latin1]{inputenc}
\usepackage[T1]{fontenc}

% change the alerted colour to LimeGreen
\definecolor{LimeGreen}{RGB}{50,205,50}
\setbeamercolor{structure}{fg=LimeGreen}
\author[FOSSEE]{}
\institute[IIT Bombay]{}
\date[]{}
% \setbeamercovered{transparent}

% theme split

\newenvironment{colorverbatim}[1][]%
{%
\color{blue}
\verbatim
}%
{%
\endverbatim
}%
\usepackage{verbatim}
\usepackage{mathpazo,courier,euler}
\usepackage{listings}
\lstset{language=sh,
    basicstyle=\ttfamily\bfseries,
  showstringspaces=false,
  keywordstyle=\color{black}\bfseries}
\newcommand{\typ}[1]{\lstinline{#1}}
%\newcommand{\kwrd}[1]{ \texttt{\textbf{\color{blue}{#1}}}  }
% logo
\logo{\includegraphics[height=1.30 cm]{../images/3t-logo.pdf}}
\logo{\includegraphics[height=1.30 cm]{../images/fossee-logo.pdf}

\hspace{7.5cm}
\includegraphics[scale=0.99]{../images/fossee-logo.pdf}\\
\hspace{281pt}
\includegraphics[scale=0.80]{../images/3t-logo.pdf}}
%%%%%%%%%%%%%%%%%%%%%%%%%%%%%%%%%%%%%%%%%%%%%%%%%%%%%%%%%%%%%%%%%%%%%%
% DOCUMENT STARTS
\begin{document}

\sffamily \bfseries
\title
[{\LaTeX}: Basics and Structure]
{{\LaTeX}: Basics and Structure}
\author
[FOSSEE]
{\small Talk to a Teacher\\{\color{blue}\url{http://spoken-tutorial.org}}
\\\vspace{0.25cm}National Mission on Education
 through ICT\\{\color{blue}\url{ http://sakshat.ac.in}} \\ [1.65cm]
   Contributed by FOSSEE Team \\IIT Bombay  \\[0.3cm]
}

% slide 1
\begin{frame}
   \titlepage
\end{frame}

\begin{frame}
  \frametitle{Objectives}
  \label{sec-2}
    At the end of this tutorial, you will be able to,
  \begin{itemize}
    \item Understand basic structure of a LaTeX document, 
          its various document classes and loading packages 
    \item Create a LaTeX document with a title and an abstract
  \end{itemize}
\end{frame}

\begin{frame}
  \frametitle{Objectives..}
  \begin{itemize}
    \item Create numbered and non-numbered sections and subsections
          in a LaTeX document
    \item Create an appendix in a LaTeX document
    \item Create a table of content in a LaTeX document
  \end{itemize}
\end{frame}


\begin{frame}
\frametitle{Pre-requisite}
\label{sec-3}

  Spoken tutorial on,

\begin{itemize}
\item {\LaTeX} Introduction
\end{itemize}
\end{frame}


\begin{frame}[fragile]
  \frametitle{A Very Basic {\LaTeX} document}
  {\tiny
    \begin{verbatim}
      \documentclass{article}
        \begin{document}
          SciPy is open-source software for mathematics, 
           science, and engineering.
        \end{document}
    \end{verbatim}
  }
\end{frame}


\begin{frame}[fragile]
  \frametitle{\typ{documentclass command}}
  \begin{itemize}
    \item Used to select the \emph{class} of our document
    \item Some available classes - \typ{article}, \typ{proc},
      \typ{report}, \typ{book}, \typ{slides}, \typ{letter}
  \end{itemize}
\end{frame}

\begin{frame}[fragile]
  \frametitle{\typ{documentclass command}..}
    \begin{itemize}
    \item Example:
      \typ{\\documentclass}
       \typ{\[12pt,a4paper,draft\]\{report\}}\\
      \tiny The parameters within \typ{\[ \]} are optional.
      \begin{itemize}
        \item \typ{12pt} -- sets the font size of main font 
        \item \typ{a4paper} -- specify paper size
        \item \typ{draft} -- makes LaTeX indicate hyphenation and justification
                  problems
      \end{itemize}
  \end{itemize}
\end{frame}


\begin{frame}[fragile]
  \frametitle{\typ{usepackage command}}
  \begin{center}
      \lstinline{\usepackage[options]}\{...\}
  \end{center}
\end{frame}

\begin{frame}[fragile]
  \frametitle{\typ{Top Matter}}
    Adding Title, Author name \& Date to our document,
    \begin{itemize}
    \item \lstinline$\title{}$, to add title
    \item \lstinline$\author{}$, to add author name
    \item \lstinline$\date{}$, to insert todays date
    \item Compile
    \item Nothing changes
  \end{itemize}
\end{frame}

\begin{frame}[fragile]
  \frametitle{\typ{Top Matter}..}
  Now we add \alert{maketitle} command, which inserts the top-matter
  {\tiny
    \begin{verbatim}
      \documentclass{article}
      \title{A Glimpse at Scipy}
      \author{FOSSEE}
      \date{June 2010}
      \begin{document}
        \maketitle
        SciPy is open-source software for mathematics, science,
        and engineering.
      \end{document}
    \end{verbatim}
  }
\end{frame}

\begin{frame}[fragile]
  \frametitle{\typ{abstract command}}
  \begin{itemize}
    \item \typ{abstract} environment inserts abstract
    \item \lstinline$\begin{abstract}$
    \item \lstinline$\end{abstract}$
    \item Place it at the location where you want your abstract
  \end{itemize}
\end{frame}

\begin{frame}[fragile]
  \frametitle{\typ{abstract command}..}
{\tiny 
    \begin{verbatim}
      \documentclass{article}
      \title{A Glimpse at Scipy}
      \author{FOSSEE}
      \date{June 2010}
      \begin{document}
        \maketitle
        \begin{abstract}
          This document shows a glimpse of the features of Scipy.
        \end{abstract}
        SciPy is open-source software for mathematics, science, 
        and engineering.
      \end{document}
    \end{verbatim}
  }
\end{frame}

\begin{frame}[fragile]
  \frametitle{\typ{Sectioning}}
  \begin{itemize}
    \item \lstinline{\section}, \lstinline{\subsection}
      \lstinline{\subsubsection}
    \item Auto numbered sections!
    \item \typ{*} to prevent numbering of a section
  \end{itemize}
\end{frame}

\begin{frame}[fragile]
  \frametitle{\typ{Sectioning}..}
  {\tiny 
  \begin{verbatim}
     \documentclass{article}
     \title{Sectioning}
     \author{FOSSEE}
     \date{31-February-2012}
     \begin{document}
       \maketitle
       Hello world!
       \section{Numbered Section 1}
         Section1 Text
       \section{Numbered Section 2}
         Section2 Text
       \section*{Unnumbered Section 1}
         Section3 Text
       \section*{Unnumbered Section 2}
         Section4 Text
     \end{document}
  \end{verbatim}
  }
\end{frame}


\begin{frame}[fragile]
  \frametitle{\typ{Creating Chapters}}
  \begin{itemize}
    \item Longer documents, use \lstinline{report} or \lstinline{book}
    class
    \item Chapter can be added using \lstinline{\chapter} command
    \item Books \alert{do not} have the abstract environment.
  \end{itemize}
\end{frame}

\begin{frame}[fragile]
  \frametitle{\typ{Creating Chapters}..}
  {\tiny 
  \begin{verbatim}
    \documentclass{book}
      \title{My first book}
      \begin{document}
        \chapter{My First Chapter}
          Main
        \section{Section1}
          Section 1 Text
        \subsubsection{My First Subsection}
          Numbered-Section 1's Subsection Text
        \section{Section2}
          Numbered-Section 2 Text
        \section*{Section3}
          First un-numbered Section Text
         \section*{Section4}
          Second un-numbered Section Text
         \chapter{So We say goodbye}
           Thank you for reading dear reader
      \end{document}
  \end{verbatim}
  }
\end{frame}


\begin{frame}[fragile]
  \frametitle{Appendices}
  \begin{itemize}
    \item \lstinline{\appendix} command indicates the beginning of appendices.
    \item Any content after \lstinline{\appendix}, will be added to the appendix.
    \item Use sectioning commands to add sections.
  \end{itemize}
\end{frame}


\begin{frame}[fragile]
  \frametitle{\typ{Table of Contents [TOC]}}
  \begin{itemize}
    \item Add \lstinline{\tableofcontents} where you want TOC to
      appear
    \item Compile
    \item Only headings appear. No page numbers
    \item A \lstinline{.toc} file is generated
    \item Re-compile
  \end{itemize}
\end{frame}

\begin{frame}[fragile]
  \frametitle{\typ{TOC}..}
  \begin{itemize}
  \item Any numbered section/block automatically appears
  \item Un-numbered sections are added to TOC using
    \lstinline{\addcontentsline}
  \item For instance,
  \lstinline+\addcontentsline{toc}{section}+
  \lstinline+{intro}+
  \end{itemize}
\end{frame}

\begin{frame}[fragile]
  \frametitle{\typ{Exercise 1}}
  \begin{center}
    Write a LaTeX script that creates a document of type book,
    containing both a TOC at the beginning and an appendix at the end 
    of document. The book should contain two chapters with numbered \& 
    un-numbered sections \& these chapters should appear in the TOC.
  \end{center}
\end{frame}

\begin{frame}[fragile]
  \frametitle{\typ{Excercise 1: Solution}}
  {\tiny 
   \begin{center}
    \begin{verbatim}
    \documentclass{book}
      \title{My first book}
      \begin{document}
        \tableofcontents
        \addcontentsline{toc}{section}{Chapter}
        \chapter{My First Chapter}
          Main
        \section{Section1}
          Section 1 Text
        \subsubsection{My First Subsection}
          Numbered-Section 1's Subsection Text
            \section*{Section3}
          First un-numbered Section Text
         \chapter{So We say goodbye}
           Thank you for reading dear reader
           \section*{Section3}
          First un-numbered Section Text
        \appendix
          \section{Appendix - 1}
      \end{document}
    \end{verbatim}
  \end{center}
  }
\end{frame}


\begin{frame}[fragile]
  \frametitle{\typ{Summary}}
  \label{sec-8}
  In this tutorial, we have,
  \begin{itemize}
    \item Understood the basic structure of a LaTeX document, 
    its various document classes and loading packages that add new features to 
    the LaTeX system
    \item Created a LaTeX document with a title and an abstract.
  \end{itemize}
\end{frame}

\begin{frame}[fragile]
  \frametitle{\typ{Summary}..}
  \begin{itemize}
    \item Created both numbered and non-numbered sections and subsections in a 
    LaTeX document
    \item Created an appendix in a LaTeX document
    \item Created a table of content in a LaTeX document
  \end{itemize}
\end{frame}


\begin{frame}[fragile]
  \frametitle{\typ{Self assessment}}
  \label{sec-9}
  {\footnotesize 
  \begin{enumerate}
    \item  Is the LaTeX code given below a valid input file 
    (File compiles successfully and produces the intended result, 
    that is to produce a book with two chapters and an appendix)
  {\tiny 
    \begin{center}
      \begin{verbatim}
        \documentclass{book}
        \title{My book}
        \author{FOSSEE}
        \date{31-February-2012}
        \begin{document}
          \maketitle
          \chapter{My First Chapter}
          Main
          \chapter{So We say goodbye}
          Thank you for reading dear reader
          \appendix
            \section{First Appendix}
        \end{document}
      \end{verbatim}
    \end{center}
    }
    \end{enumerate}
  }
\end{frame}

\begin{frame}[fragile]
\frametitle{\typ{Solution}}
\label{sec-10}
\begin{enumerate}
\item Although the given file looks syntactically valid, the output file is not
    what we expected. This is mainly because we are trying to use the section
    command to create sections in the appendix, for a document whose type is
    given as a book.
\vspace{15pt}
\end{enumerate}
\end{frame}

\begin{frame}
\frametitle{SDES \& FOSSEE}
\begin{center}
\begin{itemize}
\item \small{SDES}\\
\small{\color{LimeGreen}Software Development techniques for 
Engineers and Scientists} \\
\scriptsize An initiative by FOSSEE. \\
\vspace{3pt}
\scriptsize For more information on SDES, please visit 
{\color{blue}\url{http://fossee.in/sdes}}\\
\vspace{12pt}
\item \small{FOSSEE}\\
\small {\color{LimeGreen}Free and Open-source Software for \\
Science and Engineering Education} \\
\scriptsize Based at IIT Bombay, Funded by MHRD.\\
\vspace{3pt}
\scriptsize Part of National Mission on Education through ICT (NME-ICT) \\
\end{itemize}
\end{center}
\end{frame}

\begin{frame}
\frametitle{About the Spoken Tutorial Project}
\begin{itemize}
\item Watch the video available at {\color{blue}\url{http://spoken-tutorial.org
 /What\_is\_a\_Spoken\_Tutorial}} 
\item It summarises the Spoken Tutorial project 
\item If you do not have good bandwidth, you can download and watch it
\end{itemize}
\end{frame}

\begin{frame}
\frametitle{Spoken Tutorial Workshops}The Spoken Tutorial Project Team 
\begin{itemize}
\item Conducts workshops using spoken tutorials 
\item Gives certificates to those who pass an online test 
\item For more details, please write to \\ \hspace {0.5cm}
{\color{blue}contact@spoken-tutorial.org}
\end{itemize}
\end{frame}

\begin{frame}
\frametitle{Acknowledgements}
\begin{itemize}
\item Spoken Tutorial Project is a part of the Talk to a Teacher  project 
\item It is supported by the National Mission on Education through  ICT, MHRD, 
Government of India 
\item More information on this Mission is available at: \\
{\color{blue}\url{http://spoken-tutorial.org/NMEICT-Intro}}
\end{itemize}
\end{frame}

\begin{frame}
  \begin{block}{}
  \begin{center}
  {\Large THANK YOU!} 
  \end{center}
  \end{block}
\begin{block}{}
  \begin{center}
    For more Information, visit our website\\
    {\color{blue}\url{http://fossee.in/}}
  \end{center}  
  \end{block}
\end{frame}
\end{document}



