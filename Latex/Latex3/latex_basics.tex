\documentclass[17pt,compress]{beamer}
\usepackage{beamerthemesplit}
\mode<presentation>
{
  \usetheme{Warsaw}
  \useoutertheme{infolines}
  \setbeamercovered{transparent}
  \setbeamertemplate{navigation symbols}{}
}
% Taken from Fernando's slides.
\usepackage{ae,aecompl}
\usepackage[scaled=.95]{helvet}

\usepackage[english]{babel}
%\usepackage[latin1]{inputenc}
\usepackage[T1]{fontenc}

% change the alerted colour to LimeGreen
\definecolor{LimeGreen}{RGB}{50,205,50}
\setbeamercolor{structure}{fg=LimeGreen}
\author[FOSSEE]{}
\institute[IIT Bombay]{}
\date[]{}
% \setbeamercovered{transparent}

% theme split

\newenvironment{colorverbatim}[1][]%
{%
\color{blue}
\verbatim
}%
{%
\endverbatim
}%
\usepackage{verbatim}
\usepackage{mathpazo,courier,euler}
\usepackage{listings}
\lstset{language=sh,
    basicstyle=\ttfamily\bfseries,
  showstringspaces=false,
  keywordstyle=\color{black}\bfseries}
\newcommand{\typ}[1]{\lstinline{#1}}
%\newcommand{\kwrd}[1]{ \texttt{\textbf{\color{blue}{#1}}}  }
% logo
\logo{\includegraphics[height=1.30 cm]{../images/3t-logo.pdf}}
\logo{\includegraphics[height=1.30 cm]{../images/fossee-logo.pdf}

\hspace{7.5cm}
\includegraphics[scale=0.99]{../images/fossee-logo.pdf}\\
\hspace{281pt}
\includegraphics[scale=0.80]{../images/3t-logo.pdf}}
%%%%%%%%%%%%%%%%%%%%%%%%%%%%%%%%%%%%%%%%%%%%%%%%%%%%%%%%%%%%%%%%%%%%%%
% DOCUMENT STARTS
\begin{document}

\sffamily \bfseries
\title
[{\LaTeX}: Basics and Structure]
{{\LaTeX}: Basics and Structure}
\author
[FOSSEE]
{\small Talk to a Teacher\\{\color{blue}\url{http://spoken-tutorial.org}}
\\\vspace{0.25cm}National Mission on Education
 through ICT\\{\color{blue}\url{ http://sakshat.ac.in}} \\ [1.65cm]
   Contributed by FOSSEE Team \\IIT Bombay  \\[0.3cm]
}

% slide 1
\begin{frame}
   \titlepage
\end{frame}

\begin{frame}
  \frametitle{Objectives}
  \label{sec-2}
    At the end of this tutorial, you will be able to,
  \begin{itemize}
    \item Understand basic structure of a LaTeX document, 
          its various document classes and loading packages 
    \item Create a LaTeX document with a title and an abstract
  \end{itemize}
\end{frame}

\begin{frame}
  \frametitle{Objectives..}
  \begin{itemize}
    \item Create numbered and non-numbered sections and subsections
          in a LaTeX document
    \item Create an appendix in a LaTeX document
    \item Create a table of content in a LaTeX document
  \end{itemize}
\end{frame}


\begin{frame}
\frametitle{Pre-requisite}
\label{sec-3}

  Spoken tutorial on,

\begin{itemize}
\item {\LaTeX} Introduction
\end{itemize}
\end{frame}


\begin{frame}[fragile]
  \frametitle{A Very Basic {\LaTeX} document}
  {\tiny
    \begin{verbatim}
      \documentclass{article}
        \begin{document}
          SciPy is open-source software for mathematics, 
           science, and engineering.
        \end{document}
    \end{verbatim}
  }
\end{frame}


\begin{frame}[fragile]
  \frametitle{\typ{documentclass command}}
  \begin{itemize}
    \item Used to select the \emph{class} of our document
    \item Some available classes - \typ{article}, \typ{proc},
      \typ{report}, \typ{book}, \typ{slides}, \typ{letter}
  \end{itemize}
\end{frame}

\begin{frame}[fragile]
  \frametitle{\typ{documentclass command}..}
    \begin{itemize}
    \item Example:
      \typ{\\documentclass}
       \typ{\[12pt,a4paper,draft\]\{report\}}\\
      \tiny The parameters within \typ{\[ \]} are optional.
      \begin{itemize}
        \item \typ{12pt} -- sets the font size of main font 
        \item \typ{a4paper} -- specify paper size
        \item \typ{draft} -- makes LaTeX indicate hyphenation and justification
                  problems
      \end{itemize}
  \end{itemize}
\end{frame}


\begin{frame}[fragile]
  \frametitle{\typ{usepackage command}}
  \begin{center}
      \lstinline{\usepackage[options]}\{...\}
  \end{center}
\end{frame}

\begin{frame}[fragile]
  \frametitle{\typ{Top Matter}}
  Let's add the title, author's name and the date.
  \begin{itemize}
    \item Add title, author and date. 
    \item Compile. 
    \item Nothing changes.
    \item use \lstinline$\date{}$, to insert todays date.
  \end{itemize}
  {\tiny
    \begin{verbatim}
      \documentclass{article}
      \title{A Glimpse at Scipy}
      \author{FOSSEE}
      \date{June 2010}
      \begin{document}
        SciPy is open-source software for mathematics, science, and engineering.
      \end{document}
    \end{verbatim}
  }
\end{frame}

\begin{frame}[fragile]
  \frametitle{\typ{Top Matter}\ \ldots}
  Now we add \alert{maketitle} command, which inserts the top-matter
  {\tiny
    \begin{verbatim}
      \documentclass{article}
      \title{A Glimpse at Scipy}
      \author{FOSSEE}
      \date{June 2010}
      \begin{document}
        \maketitle
        SciPy is open-source software for mathematics, science, and engineering.
      \end{document}
    \end{verbatim}
  }
\end{frame}

\begin{frame}[fragile]
  \frametitle{\typ{abstract command}}
  \begin{itemize}
    \item \typ{abstract} environment inserts abstract.
    \item Place it at the location where you want your abstract. 
  \end{itemize}
  {\tiny 
    \begin{verbatim}
      \documentclass{article}
      \title{A Glimpse at Scipy}
      \author{FOSSEE}
      \date{June 2010}
      \begin{document}
        \maketitle
        \begin{abstract}
          This document shows a glimpse of the features of Scipy that will be explored during this course.
        \end{abstract}
        SciPy is open-source software for mathematics, science, and engineering.
      \end{document}
    \end{verbatim}
  }
\end{frame}

\begin{frame}[fragile]
  \frametitle{\typ{Sectioning}}
  \begin{itemize}
    \item \lstinline{\section}, \lstinline{\subsection}
      \lstinline{\subsubsection}
    \item Auto numbered sections!
    \item \typ{*} to prevent numbering of a section
  \end{itemize}
  {\tiny 
  \begin{verbatim}
     \documentclass{article}
     \author{FOSSEE}
     \date{31-February-2012}
     \begin{document}
       \maketitle
       Hello world!
       \section{Numbered Section 1}
         Section1 Text
       \section{Numbered Section 2}
         Section2 Text
       \section*{Unnumbered Section 1}
         Section3 Text
       \section*{Unnumbered Section 2}
         Section4 Text
     \end{document}
  \end{verbatim}
  }
\end{frame}

\begin{frame}[fragile]
  \frametitle{\typ{Creating Chapters}}
  \begin{itemize}
    \item Longer documents, use \lstinline{report} or \lstinline{book}
    class
    \item Chapter can be added using \lstinline{\chapter} command
    \item Books \alert{do not} have the abstract environment.
  \end{itemize}
  {\tiny 
  \begin{verbatim}
    \documentclass{book}
      \title{My first book}
      \begin{document}
        \chapter{My First Chapter}
          Main
        \section{Section1}
          Section 1 Text
        \subsubsection{My First Subsection}
          Numbered-Section 1's Subsection Text
        \section{Section2}
          Numbered-Section 2 Text
        \section*{Section3}
          First un-numbered Section Text
         \section*{Section4}
          Second un-numbered Section Text
         \chapter{So We say goodbye}
           Thank you for reading dear reader
      \end{document}
  \end{verbatim}
  }
\end{frame}

\begin{frame}[fragile]
  \frametitle{\typ{Sectioning and numbering}}
  \begin{itemize}
  \item subsections do not get numbering
  \item Change \lstinline{secnumdepth}
  \end{itemize}
  \begin{lstlisting}
      \setcounter{secnumdepth}{3}
  \end{lstlisting}
\end{frame}

\begin{frame}[fragile]
  \frametitle{Appendices}
  \begin{itemize}
    \item \lstinline{\appendix} command indicates the beginning of appendices.
    \item Any content after \lstinline{\appendix}, will be added to the appendix.
    \item Use sectioning commands to add sections.
  \end{itemize}
\end{frame}


\begin{frame}[fragile]
  \frametitle{\typ{Table of Contents [TOC]}}
  \begin{itemize}
    \item Our document is short, but let's learn to add a TOC
    \item Add \lstinline{\tableofcontents} where you want TOC to
      appear
    \item Compile
    \item Only headings appear. No page numbers
    \item A \lstinline{.toc} file is generated
    \item Re-compile
    \item Any numbered section/block automatically appears
  \end{itemize}
\end{frame}

\begin{frame}[fragile]
  \frametitle{\typ{TOC}\ \ldots}
  \begin{itemize}
  \item Un-numbered sections are added to TOC using
    \lstinline{\addcontentsline}
  \item For instance,  \lstinline+\addcontentsline{toc}{section}{Intro}+
  \end{itemize}
\end{frame}

\begin{frame}[fragile]
  \frametitle{\typ{Exercise 1}}
  \begin{center}
    Write a LaTeX script that creates a document of type article, which contains both a table of content and an appendix. The table of content should be at the beginning of the document and the appendix at the end.
    The book should contain two chapters, with the first chapter containing two numbered and two un-numbered sections. The first un-numbered section should be present in the table of content.
  \end{center}
\end{frame}

\begin{frame}[fragile]
  \frametitle{\typ{Excercise 1: Solution}}
  \begin{center}
     Note - This File needs to be compiled twice
  \end{center}
  {\tiny 
   \begin{center}
    \begin{verbatim}
      \documentclass{article}
         \title{article with an appendix}
         \begin{document}
         \tableofcontents
         \pagebreak
         Main content 
         \section{Numbered Section1}
           Section 1 Text
         \subsection{Numbered  Subsection1}
           Numbered-Section 1’s Subsection Text.
         \section{Numbered Section2}
           Numbered-Section 2 Text
         \section*{Un-numbered Section3}
           \addcontentsline{toc}{section}{Numbered Subsection1}
           First un-numbered Section Text.\\This appears in the table of content
         \section*{Un-numbered Section4}
           Second un-numbered Section Text
         \appendix
          \section{Appendix - 1}
      \end{document}
    \end{verbatim}
  \end{center}
  }
\end{frame}


\begin{frame}[fragile]
  \frametitle{\typ{Summary}}
  \label{sec-8}
  In this tutorial, we have,
  \begin{itemize}
    \item Gained an understanding of the basic structure of a LaTeX document, its various document classes and loading packages that add new features to 
    \item   the LaTeX system.
    \item Created a LaTeX document with a title and an abstract.
    \item Created both numbered and non-numbered sections and subsections in a LaTeX document.
    \item Created an appendix in a LaTeX document.
    \item Created a table of content in a LaTeX document.
  \end{itemize}
\end{frame}

\begin{frame}[fragile]
  \frametitle{\typ{Self assessment questions}}
  \label{sec-9}
  {\footnotesize 
  \begin{enumerate}
    \item  Is the LaTeX code given below a valid input file (File compiles successfully and produces the intended result, that is to produce a book with two chapters and an appendix.
  {\tiny 
    \begin{center}
      \begin{verbatim}
        \documentclass{book}
        \title{My book}
        \author{FOSSEE}
        \date{31-February-2012}
        \begin{document}
          \maketitle
          \chapter{My First Chapter}
          Main
          \chapter{So We say goodbye}
          Thank you for reading dear reader
          \appendix
            \section{First Appendix}
        \end{document}
      \end{verbatim}
    \end{center}
    }
    \item subsection command can be placed at any arbitrary level. If they get numbered by default using the appropriate setcounter command and secnumdepth parameter, do they automatically appear in the table of content ?? 
  \end{enumerate}
  }
\end{frame}

\begin{frame}[fragile]
\frametitle{\typ{Solutions}}
\label{sec-10}
\begin{enumerate}
\item Although the given file looks syntactically valid, the output file is not what we expected. This is mainly because we are trying to use the section command to create sections in the appendix, for a document whose type is given as a book.
\vspace{15pt}
\item No, the \textbackslash tableofcontents  command normally shows only numbered section headings, and only down to the level defined by the tocdepth counter.
\end{enumerate}
\end{frame}

\begin{frame}

  \begin{block}{}
  \begin{center}
  {\Large THANK YOU!} 
  \end{center}
  \end{block}
\begin{block}{}
  \begin{center}
    For more Information, visit our website\\
    {\color{blue}\url{http://fossee.in/}}
  \end{center}  
  \end{block}
\end{frame}

\end{document}



