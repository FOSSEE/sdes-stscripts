%%%%%%%%%%%%%%%%%%%%%%%%%%%%%%%%%%%%%%%%%%%%%%%%%%%%%%%%%%%%%%%%%%%%%%%%%%%%%%%%
% Introduction to LaTeX 
%
% Author: FOSSEE 
% Copyright (c) 2009, FOSSEE, IIT Bombay
%%%%%%%%%%%%%%%%%%%%%%%%%%%%%%%%%%%%%%%%%%%%%%%%%%%%%%%%%%%%%%%%%%%%%%%%%%%%%%%%

\documentclass[12pt,compress]{beamer}

\mode<presentation>
{
  \usetheme{Warsaw}
  \useoutertheme{infolines}
  \setbeamercovered{transparent}
}

\usepackage[english]{babel}
\usepackage[latin1]{inputenc}
%\usepackage{times}
\usepackage[T1]{fontenc}

% Taken from Fernando's slides.
\usepackage{ae,aecompl}
\usepackage{mathpazo,courier,euler}
\usepackage[scaled=.95]{helvet}

\definecolor{darkgreen}{rgb}{0,0.5,0}

\usepackage{listings}
\lstset{language=sh,
    basicstyle=\ttfamily\bfseries,
    commentstyle=\color{red}\itshape,
  stringstyle=\color{darkgreen},
  showstringspaces=false,
  keywordstyle=\color{blue}\bfseries}
\newcommand{\inctime}[1]{\addtocounter{time}{#1}{\tiny \thetime\ m}}

\newcommand{\typ}[1]{\lstinline{#1}}

\newcommand{\kwrd}[1]{ \texttt{\textbf{\color{blue}{#1}}}  }
\title {Introduction to {\LaTeX}}
\author {FOSSEE}
%%%%%%%%%%%%%%%%%%%%%%%%%%%%%%%%%%%%%%%%%%%%%%%%%%%%%%%%%%%%%%%%%%%%%%
% DOCUMENT STARTS
\begin{document}

\begin{frame}

\begin{center}
\vspace{12pt}
\textcolor{blue}{\huge Introduction to {\LaTeX}}
\end{center}
\vspace{18pt}
\begin{center}
\vspace{10pt}
\includegraphics[scale=0.95]{../images/fossee-logo.png}\\
\vspace{5pt}
\scriptsize Developed by FOSSEE Team, IIT-Bombay. \\ 
\scriptsize Funded by National Mission on Education through ICT\\
\scriptsize  MHRD,Govt. of India\\
\includegraphics[scale=0.30]{../images/iitb-logo.png}\\
\end{center}
\end{frame}

\begin{frame}
\frametitle{Objectives}
\label{sec-2}

At the end of this tutorial, you will be able to,
\begin{itemize}
\item Get acquainted to LaTeX.
\item Know why we prefer LaTeX ??
\item Know about the advantages and disadvantages of typesetting documents using the LaTeX approach.
\item Have a description, of a typical work flow; which uses LaTeX to typeset documents.
\item Recognize and differentiate between LaTeX commands, LaTeX comments and special characters, spacing and actual document content.
\item Create and compile a very simple LaTeX document.
\end{itemize}
\end{frame}

\begin{frame}
\frametitle{Pre-requisite}
\label{sec-3}

  Spoken tutorial on -

\begin{itemize}
\item Installing {\LaTeX}.
\end{itemize}
\end{frame}

\begin{frame}[fragile]
  \frametitle{Introduction}
  \begin{block}{{\LaTeX} - Introduction}
    \begin{itemize}
      \item Typesetting program
       \begin{itemize}
         \item What is typesetting?
       \end{itemize}
      \item Excellently Typeset Documents - specially Math
      \item Anything from one page articles to huge books
      \item Pronounced \emph{Lah-tech} or \emph{Lay-tech}
    \end{itemize}
  \end{block}
\end{frame}

\begin{frame}[fragile]
  \frametitle{Why {\LaTeX}?}
  \begin{itemize}
  \item Excellent visual quality! 
  \item Handles the typesetting; Lets you focus on content
  \item Makes writing math extremely simple
  \item It is a standard -- widely used in Scientific community
  \end{itemize}
  \begin{block}{}
    \[\tilde{N}_{\mathbf{x}}\times \mathbf{r}(\mathbf{x}) f_{1k}(\mathbf{x},t) - \frac{1}{2} \tilde{N} \tilde{N}:\mathbf{BB}^{T}P(\mathbf{x},t) = -m_{k}f_{1k}(\mathbf{x},t) + 2 \mathop{\mathbf{\aa}}_{j=1}^{K} f_{1j}(\mathbf{x},t)m_{j}P_{k|j} \]
  \end{block}
\end{frame}

\begin{frame}[fragile]
  \frametitle{Why \LaTeX? \ldots}
  \begin{itemize}
  \item {\LaTeX} is a document based mark-up
  \item Mark-up $\rightarrow$ a system of annotating text, adding extra
    information to specify structure and presentation of text
  \item Document based markup $\rightarrow$ you don't have to worry
    about each element individually 
  \item Allows you to focus on content, rather than appearance.
  \end{itemize}
\end{frame}

\begin{frame}[fragile]
  \frametitle{Advantages of using \LaTeX }
  \begin{itemize}
    \item Easy availablity of professional templates.
    \item Typesetting complex formulae in a convenient environment.
    \item Can start typesetting with very little effort.
    \item Presence of a lot of add-on packages.
    \item Encourages creation of well structured texts.
  \end{itemize}
\end{frame}

\begin{frame}[fragile]
  \frametitle{Disadvantages of using \LaTeX }
  \begin{itemize}
    \item Designing whole new layout is difficult.
    \item LaTeX is not a word processor.
  \end{itemize}
\end{frame}

\begin{frame}[fragile]
  \frametitle{\LaTeX \ input file format}
  \begin{itemize}
    \item \LaTeX takes ASCII text file as input.
    \item We can compile \LaTeX files into DVI,Postscript or PDF files.
    \item \alert{Note:} \typ{latex} vs. \typ{pdflatex} 
  \end{itemize}
\end{frame}

\begin{frame}[fragile]
  \frametitle{Commands,Comments\&Special Characters }
  \begin{itemize}
    \item {\LaTeX} is case sensetive.
    \item Commands begin with a \typ{\\}
    \item Environments have a \typ{\\begin} and \typ{\\end} 
    \item Any content after the \typ{\\end\{document\}} is ignored
  \end{itemize}
\end{frame}

\begin{frame}[fragile]
  \frametitle{Commands,Comments\&Special Characters \ldots}
  \begin{itemize}
    \item Anything that follows a \typ{\%} symbol till end of the line
      is a comment 
    \item Special characters (\typ{\~ \# \$ \^ \& \_ \{ \}}) are escaped by a
      \typ{\\} 
    \item \typ{\\} symbol is inserted using \typ{\\textbackslash}
      command
    %\item \textbackslash newline or \textbackslash \textbackslash\ is used to insert newlines.
    \item \typ{\\newline} or \typ{\\\\} is used to insert newlines.
  \end{itemize}
\end{frame}

\begin{frame}[fragile]
  \frametitle{Typesetting a minimal document}
  Write the sample code  into the file \typ{temp.tex}
  \vspace{8pt}
  {\tiny
    \begin{verbatim}
      \documentclass{article}
        \begin{document}
          SciPy is open-source software for mathematics, science, and engineering.
        \end{document}
    \end{verbatim}
  }
\end{frame}  

\begin{frame}[fragile]
  \frametitle{Compiling to DVI }
    \begin{center}
   \alert{latex temp.tex}  
    \end{center}
\end{frame}

\begin{frame}[fragile]
  \frametitle{Compiling to PDF }
    \begin{center}
      \alert{pdflatex temp.tex}  
    \end{center}
    \begin{center}
      \em Note: Throughout this course, we shall use pdflatex to compile our documents. 
    \end{center}
\end{frame}

\begin{frame}
\frametitle{Summary}
\label{sec-8}

  In this tutorial, we have learnt,
\begin{itemize}
  \item About LaTeX.
  \item why we prefer LaTeX.
  \item about the advantages and disadvantages of typesetting documents using the LaTeX approach.
  \item A description, of a typical work flow; which uses LaTeX to typeset documents.
  \item The ability to recognize and differentiate between LaTeX commands, LaTeXcomments and special characters, spacing and actual document content.
  \item Created and compiled a very simple LaTeX document.
\end{itemize}
\end{frame}


\begin{frame}[fragile]
\frametitle{Self assessment questions}
\label{sec-9}
\begin{enumerate}
  \item Convert the temp.dvi created during the course of this tutorial to temp\_1.ps using the dvips command. Verify that the two files indeed look the same.
  \vspace{8pt}
  \item Convert the temp.dvi created during the course of this tutorial to temp\_1.pdf using the dvipdfm command. Verify that the two files indeed look the same.
\end{enumerate}
\end{frame}

\begin{frame}
\frametitle{Solutions}
\label{sec-10}
\begin{enumerate}
  \item We can use the following command to convert temp.dvi to temp\_1.ps\\
    \begin{block}{}
      \begin{center}
        dvips -o temp\_1.ps temp.dvi
      \end{center}
    \end{block}
  \vspace{15pt}
  \item We can use the following command to convert temp.dvi to temp\_1.pdf\\
    \begin{block}{}
      \begin{center}
         dvipdfm -o temp\_1.pdf temp.dvi
      \end{center}
    \end{block}
\end{enumerate}
\end{frame}


\begin{frame}
  \begin{block}{}
  \begin{center}
    \textcolor{blue}{\Large THANK YOU!} 
  \end{center}
  \end{block}
\begin{block}{}
  \begin{center}
      For more Information, visit our website\\
      \url{http://fossee.in/}
  \end{center}  
  \end{block}
\end{frame}


\end{document}
