%%%%%%%%%%%%%%%%%%%%%%%%%%%%%%%%%%%%%%%%%%%%%%%%%%%%%%%%%%%%%%%%%%%%%%%%%%%%%%%%
% Introduction to LaTeX 
%
% Author: FOSSEE 
% Copyright (c) 2009, FOSSEE, IIT Bombay
%%%%%%%%%%%%%%%%%%%%%%%%%%%%%%%%%%%%%%%%%%%%%%%%%%%%%%%%%%%%%%%%%%%%%%%%%%%%%%%%

\documentclass[17pt,compress]{beamer}
\usepackage{beamerthemesplit}
\mode<presentation>
{
  \usetheme{Warsaw}
  \useoutertheme{infolines}
  \setbeamercovered{transparent}
  \setbeamertemplate{navigation symbols}{}
}
% Taken from Fernando's slides.
\usepackage{ae,aecompl}
\usepackage[scaled=.95]{helvet}

\usepackage[english]{babel}
\usepackage[latin1]{inputenc}
\usepackage[T1]{fontenc}

% change the alerted colour to LimeGreen
\definecolor{LimeGreen}{RGB}{50,205,50}
\setbeamercolor{structure}{fg=LimeGreen}
\author[FOSSEE]{}
\institute[IIT Bombay]{}
\date[]{}
% \setbeamercovered{transparent}

% theme split
\usepackage{verbatim}
\newenvironment{colorverbatim}[1][]%
{%
\color{blue}
\verbatim
}%
{%
\endverbatim
}%

\usepackage{mathpazo,courier,euler}
\usepackage{listings}
\lstset{language=sh,
    basicstyle=\ttfamily\bfseries,
  showstringspaces=false,
  keywordstyle=\color{black}\bfseries}

% logo
\logo{\includegraphics[height=1.30 cm]{../images/3t-logo.pdf}}
\logo{\includegraphics[height=1.30 cm]{../images/fossee-logo.pdf}

\hspace{7.5cm}
\includegraphics[scale=0.99]{../images/fossee-logo.pdf}\\
\hspace{281pt}
\includegraphics[scale=0.80]{../images/3t-logo.pdf}}
\newcommand{\typ}[1]{\lstinline{#1}}


\begin{document}

\sffamily \bfseries
\title
[Introduction to \LaTeX]
{Introduction to \LaTeX}
\author
[FOSSEE]
{\small Talk to a Teacher\\{\color{blue}\url{http://spoken-tutorial.org}}\\\vspace{0.25cm}National Mission on Education
 through ICT\\{\color{blue}\url{ http://sakshat.ac.in}} \\ [1.65cm]
   Contributed by FOSSEE Team \\IIT Bombay  \\[0.3cm]
}

% slide 1
\begin{frame}
   \titlepage
\end{frame}

\begin{frame}
\frametitle{Objectives}
\label{sec-2}

At the end of this tutorial, you will,
\begin{itemize}
\item Get acquainted to LaTeX.
\item Know why we prefer LaTeX.
\item Know the advantages and disadvantages of typesetting documents in LaTeX.
\end{itemize}
\end{frame}

\begin{frame}
\frametitle{Objectives..}
\label{sec-2}

\begin{itemize}
\item Get a brief idea on typical work flow that uses LaTeX to typeset
documents.
\item Know about LaTeX commands, comments,
      special characters, spacing, actual content.
\item Be able to create and compile a simple LaTeX document.
\end{itemize}
\end{frame}

\begin{frame}
\frametitle{Pre-requisite}
\label{sec-3}

  Spoken tutorial on - 

\begin{itemize}
\item {\LaTeX} Installation.
\end{itemize}
\end{frame}

\begin{frame}[fragile]
  \frametitle{Introduction}
    \begin{itemize}
     \item Donald E. Knuth.
      \item Typesetting program.
      \item Excellently Typeset Documents - specially Math.
      \item Anything from one page articles to huge books.
      \item Pronounced \emph{Lah-tech} or \emph{Lay-tech}.
    \end{itemize}
\end{frame}

\begin{frame}[fragile]
  \frametitle{Why {\LaTeX}?}
  \begin{itemize}
  \item Excellent visual quality. 
  \item Handles typesetting; lets you focus on content.
  \item Makes writing Math extremely simple.
  \item It is a standard -- widely used in Scientific community.
  \end{itemize}
    \[\tilde{N}_{\mathbf{x}}\times \mathbf{r}(\mathbf{x}) f_{1k} \]
\end{frame}

\begin{frame}[fragile]
  \frametitle{Why \LaTeX?..}
  \begin{itemize}
  \item It is a document based mark-up.
  \item Mark-up: annotating text, adding
    information to specify structure and presentation of text.
  \item Document based markup: don't have to worry
    about individual elements. 
  \item Allows you to focus on content.
  \end{itemize}
\end{frame}

\begin{frame}[fragile]
  \frametitle{Advantages of using \LaTeX }
  \begin{itemize}
    \item Availability of professional templates.
    \item Typesetting complex formulae in a convenient environment.
    \item Typesetting with very little effort.
    \item Lot of add-on packages available.
    \item Easy creation of well structured texts.
  \end{itemize}
\end{frame}

\begin{frame}[fragile]
  \frametitle{Disadvantages of using \LaTeX }
  \begin{itemize}
    \item Designing whole new layout is difficult.
    \item LaTeX is not a word processor.
  \end{itemize}
\end{frame}

\begin{frame}[fragile]
  \frametitle{\LaTeX \ input file format}
  \begin{itemize}
    \item \LaTeX \: takes ASCII text file as input.
    \item We can compile \LaTeX  \: files into DVI, Postscript or PDF files.
    \item \alert{Note:} \typ{latex} vs. \typ{pdflatex} 
  \end{itemize}
\end{frame}

\begin{frame}[fragile]
  \frametitle{Commands,Comments\&Special Characters }
  \begin{itemize}
    \item {\LaTeX} is case sensetive.
    \item Commands begin with a \typ{\\}
    \item Environments have a \typ{\\begin} and \typ{\\end} 
    \item Any content after the \typ{\\end\{document\}} is ignored.
  \end{itemize}
\end{frame}

\begin{frame}[fragile]
  \frametitle{Commands,Comments\&Special Characters..}
  \begin{itemize}
    \item Anything after \typ{\%} symbol till end of the line
      is a comment. 
    \item Special characters (\typ{\~ \# \$ \^ \& \_ \{ \}}) are escaped by a
      \typ{\\} 
    \item \typ{\\} is inserted using \typ{\\textbackslash},
     \typ{\\newline} or \typ{\\\\} to insert newlines.
  \end{itemize}
\end{frame}

\begin{frame}[fragile]
  \frametitle{Typesetting a minimal document}
  Write the sample code  into the file \typ{temp.tex}
  \vspace{8pt}
  {\tiny
    \begin{verbatim}
      \documentclass{article}
        \begin{document}
          SciPy is open-source software for mathematics, science, and 
          engineering.
        \end{document}
    \end{verbatim}
  }
\end{frame}  

\begin{frame}[fragile]
  \frametitle{Compiling to DVI and PDF}
    \begin{center}
   \alert{latex temp.tex} \\
    \alert{pdflatex temp.tex} \\
      \em Note: Throughout this course, we shall use pdflatex to compile our
        documents. 
    \end{center}
\end{frame}

\begin{frame}
\frametitle{Summary}
\label{sec-8}

  In this tutorial, we have learnt,
\begin{itemize}
  \item About LaTeX.
  \item why we prefer LaTeX.
  \item advantages and disadvantages of typesetting documents using
        LaTeX approach.
\end{itemize}
\end{frame}

\begin{frame}
\frametitle{Summary..}
\label{sec-8}

\begin{itemize}
  \item About a typical work flow; which uses LaTeX to typeset
        documents.
  \item About LaTeX commands, comments, special characters, spacing,
        actual document content.
  \item How to create and compile a simple LaTeX document.
\end{itemize}
\end{frame}


\begin{frame}[fragile]
\frametitle{Self assessment questions}
\label{sec-9}
\begin{enumerate}
  \item Convert temp.dvi created in the tutorial to temp\_1.ps
        using dvips command. Verify that  both files look same.
  \vspace{8pt}
  \item Convert this temp.dvi file to temp\_1.pdf using dvipdfm command.
        Verify both the files look same.
\end{enumerate}
\end{frame}

\begin{frame}
\frametitle{Solutions}
\label{sec-10}
\begin{enumerate}
  \item We can use the following command to convert temp.dvi to temp\_1.ps\\
      \begin{center}
        dvips -o temp\_1.ps temp.dvi
      \end{center}
  \vspace{15pt}
  \item We can use the following command to convert temp.dvi to temp\_1.pdf\\
      \begin{center}
        dvipdfm -o temp\_1.pdf temp.dvi
      \end{center}
\end{enumerate}
\end{frame}


\begin{frame}
\frametitle{SDES \& FOSSEE}
\begin{center}
\begin{itemize}
\item \small{SDES}\\
\small{\color{LimeGreen}Software Development techniques for Engineers and Scientists} \\
\scriptsize An initiative by FOSSEE. \\
\vspace{3pt}
\scriptsize For more information on SDES, please visit {\color{blue}\url{http://fossee.in/sdes}}\\
\vspace{12pt}
\item \small{FOSSEE}\\
\small {\color{LimeGreen}Free and Open-source Software for \\Science and Engineering Education} \\
\scriptsize Based at IIT Bombay, Funded by MHRD.\\
\vspace{3pt}
\scriptsize Part of National Mission on Education through ICT (NME-ICT). \\
\end{itemize}
\end{center}
\end{frame}

\begin{frame}
\frametitle{About the Spoken Tutorial Project}
\begin{itemize}
\item Watch the video available at {\color{blue}\url{http://spoken-tutorial.org /What\_is\_a\_Spoken\_Tutorial}} 
\item It summarises the Spoken Tutorial project 
\item If you do not have good bandwidth, you can download and watch it
\end{itemize}
\end{frame}

\begin{frame}
\frametitle{Spoken Tutorial Workshops}The Spoken Tutorial Project Team 
\begin{itemize}
\item Conducts workshops using spoken tutorials 
\item Gives certificates to those who pass an online test 
\item For more details, please write to \\ \hspace {0.5cm}{\color{blue}contact@spoken-tutorial.org}
\end{itemize}
\end{frame}

\begin{frame}
\frametitle{Acknowledgements}
\begin{itemize}
\item Spoken Tutorial Project is a part of the Talk to a Teacher  project 
\item It is supported by the National Mission on Education through  ICT, MHRD, Government of India 
\item More information on this Mission is available at: \\{\color{blue}\url{http://spoken-tutorial.org/NMEICT-Intro}}
\end{itemize}
\end{frame}

\begin{frame}
  \begin{block}{}
  \begin{center}
  {\Large THANK YOU!} 
  \end{center}
  \end{block}
\begin{block}{}
  \begin{center}
    For more Information, visit our website\\
    {\color{blue}\url{http://fossee.in/}}
  \end{center}  
  \end{block}
\end{frame}

\end{document}
