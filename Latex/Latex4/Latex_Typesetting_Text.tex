%%%%%%%%%%%%%%%%%%%%%%%%%%%%%%%%%%%%%%%%%%%%%%%%%%%%%%%%%%%%%%%%%%%%%%%%%%%%%%%%
% LateX
%
% Author: FOSSEE 
% Copyright (c) 2009, FOSSEE, IIT Bombay
%%%%%%%%%%%%%%%%%%%%%%%%%%%%%%%%%%%%%%%%%%%%%%%%%%%%%%%%%%%%%%%%%%%%%%%%%%%%%%%%

\documentclass[17pt,compress]{beamer}
\usepackage{beamerthemesplit}
\mode<presentation>
{
  \usetheme{Warsaw}
  \useoutertheme{infolines}
  \setbeamercovered{transparent}
  \setbeamertemplate{navigation symbols}{}
}
% Taken from Fernando's slides.
\usepackage{ae,aecompl}
\usepackage[scaled=.95]{helvet}

\usepackage[english]{babel}
\usepackage[latin1]{inputenc}
\usepackage[T1]{fontenc}

% change the alerted colour to LimeGreen
\definecolor{LimeGreen}{RGB}{50,205,50}
\setbeamercolor{structure}{fg=LimeGreen}
\author[FOSSEE]{}
\institute[IIT Bombay]{}
\date[]{}
% \setbeamercovered{transparent}

% theme split
\usepackage{verbatim}
\newenvironment{colorverbatim}[1][]%
{%
\color{blue}
\verbatim
}%
{%
\endverbatim
}%

\usepackage{mathpazo,courier,euler}
\usepackage{listings}
\lstset{language=sh,
    basicstyle=\ttfamily\bfseries,
  showstringspaces=false,
  keywordstyle=\color{black}\bfseries}

% logo
\logo{\includegraphics[height=1.30 cm]{../images/3t-logo.pdf}}
\logo{\includegraphics[height=1.30 cm]{../images/fossee-logo.pdf}

\hspace{7.5cm}
\includegraphics[scale=0.99]{../images/fossee-logo.pdf}\\
\hspace{281pt}
\includegraphics[scale=0.80]{../images/3t-logo.pdf}}
\newcommand{\typ}[1]{\lstinline{#1}}

\begin{document}

\sffamily \bfseries
\title
[{\LaTeX}: Typesetting Text]
{{\LaTeX}: Typesetting Text}
\author
[FOSSEE]
{\small Talk to a Teacher\\{\color{blue}\url{http://spoken-tutorial.org}}\\\vspace{0.25cm}National Mission on Education
 through ICT\\{\color{blue}\url{ http://sakshat.ac.in}} \\ [1.65cm]
   Contributed by FOSSEE Team \\IIT Bombay  \\[0.3cm]
}

\begin{frame}
   \titlepage
\end{frame}

\begin{frame}
  \frametitle{Objectives}
  At the end of this session, you will be able to:
  \begin{itemize}
  \item Learn how to typeset your document using {\LaTeX}
  \item Use lists, listings in your document for formatting text
  \end{itemize}
\end{frame}

\begin{frame}
  \frametitle{Prerequisite}
  Spoken Tutorial on:
  \begin{itemize}
  \item {\LaTeX} Part 1 - Getting Started
  \item {\LaTeX} Part 2 - Introduction
  \item {\LaTeX} Part 3 - Structuring the Content
  \end{itemize}
\end{frame}

\begin{frame}[fragile]
  \frametitle{Quotation Marks}
  \begin{itemize}
  \item Use \`~ (accent) for left quote
  \item Use \'~ (apostrophe) for right quote
  \item For double quotes, use them twice
  \end{itemize}
\end{frame}

\begin{frame}[fragile]
  \frametitle{Fonts - Emphasis, Fixed width}
  \begin{itemize}
  \item \lstinline{\emph} gives emphasized or italic text
  \item \typ{flushleft} to have text left aligned
  \item \typ{flushright}, \typ{center}
  \end{itemize}
\end{frame}

\begin{frame}[fragile]
  \frametitle{Fonts - Emphasis, Fixed width..}
  \begin{itemize}
  \item \lstinline{\texttt} gives fixed width font
  \item \lstinline{\textbf} bold face font
  \item \lstinline{--} en dash (--); \lstinline{---} em dash (---). 
  \end{itemize}
\end{frame}

\begin{frame}[fragile]
  \frametitle{Lists}
  \begin{itemize}
  \item \lstinline{enumerate} is used for numbered lists
  \item \lstinline{itemize} gives un-numbered lists
  \item New item in list is specified using \lstinline{\item}
  \item Nested lists are also easily handled
  \end{itemize}
\end{frame}

\begin{frame}[fragile]
  \frametitle{Footnotes}
  \begin{itemize}
  \item \typ{\\footnote} command adds a footnote
  \end{itemize}
\end{frame}

\begin{frame}[fragile]
  \frametitle{Labels and References}
  \begin{itemize}
  \item \lstinline+\label{labelname}+ is used to label an element
  \item \lstinline+\ref{labelname}+ is used to refer to that element
  \item Compile twice
  \end{itemize}
\end{frame}

\begin{frame}[fragile]
  \frametitle{Including code}
  \begin{itemize}
  \item We could use \lstinline{\verbatim} 
  \item \lstinline+listings+ is a powerful package
  \item \lstinline+\usepackage{listings}+ needs to be added 
  \item Spefify language either by using 
    \small\typ{\\lstinputlisting[language=Python]} or \typ{\\lstset}
  \end{itemize}
\end{frame}

\begin{frame}[fragile]
  \frametitle{Including code..}
  \begin{itemize}
  \item Use \lstinline+\lstlisting+ for a block of code
  \item \typ{\\lstinline} for inline code
  \end{itemize}
\end{frame}


\begin{frame}[fragile]
	\frametitle{Summary...}
	\begin{itemize}
	\item Put Quotation Marks around text.
	\item Emphasize and give fixed width to fonts.
	\item Use numbered and un-numbered lists.
	\item Add Footnotes, Labels and References.
	\item To include code.
	\end{itemize}
\end{frame}

\begin{frame}[fragile]
\frametitle{Evaluation}
\begin{enumerate}
\item Which environment is used to include a block of code?
\item Joe has numerous used labels inside his Latex document. 
But all the references to label names come up as question marks.
What might be the problem? 
\end{enumerate}
\end{frame}
\begin{frame}

\frametitle{Solutions}
\begin{enumerate}
\item lstlistings
\vspace{15pt}
\item While using labels, the latex document should be compiled 
twice for the references to show up.
\end{enumerate}
\end{frame}

\begin{frame}
\frametitle{SDES \& FOSSEE}
\begin{center}
\begin{itemize}
\item \small{SDES}\\
\small{\color{LimeGreen}Software Development techniques for Engineers and Scientists} \\
\scriptsize An initiative by FOSSEE. \\
\vspace{3pt}
\scriptsize For more information on SDES, please visit {\color{blue}\url{http://fossee.in/sdes}}\\
\vspace{12pt}
\item \small{FOSSEE}\\
\small {\color{LimeGreen}Free and Open-source Software for \\Science and Engineering Education} \\
\scriptsize Based at IIT Bombay, Funded by MHRD.\\
\vspace{3pt}
\scriptsize Part of National Mission on Education through ICT (NME-ICT). \\
\end{itemize}
\end{center}
\end{frame}

\begin{frame}
\frametitle{About the Spoken Tutorial Project}
\begin{itemize}
\item Watch the video available at {\color{blue}\url{http://spoken-tutorial.org /What\_is\_a\_Spoken\_Tutorial}} 
\item It summarises the Spoken Tutorial project 
\item If you do not have good bandwidth, you can download and watch it
\end{itemize}
\end{frame}

\begin{frame}
\frametitle{Spoken Tutorial Workshops}The Spoken Tutorial Project Team 
\begin{itemize}
\item Conducts workshops using spoken tutorials 
\item Gives certificates to those who pass an online test 
\item For more details, please write to \\ \hspace {0.5cm}{\color{blue}contact@spoken-tutorial.org}
\end{itemize}
\end{frame}

\begin{frame}
\frametitle{Acknowledgements}
\begin{itemize}
\item Spoken Tutorial Project is a part of the Talk to a Teacher  project 
\item It is supported by the National Mission on Education through  ICT, MHRD, Government of India 
\item More information on this Mission is available at: \\{\color{blue}\url{http://spoken-tutorial.org/NMEICT-Intro}}
\end{itemize}
\end{frame}

\begin{frame}
  \begin{block}{}
  \begin{center}
  {\Large THANK YOU!} 
  \end{center}
  \end{block}
\begin{block}{}
  \begin{center}
    For more Information, visit our website\\
    {\color{blue}\url{http://fossee.in/}}
  \end{center}  
  \end{block}
\end{frame}

\end{document}
