%%%%%%%%%%%%%%%%%%%%%%%%%%%%%%%%%%%%%%%%%%%%%%%%%%%%%%%%%%%%%%%%%%%%%%%%%%%%%%%%
% Version Control Systems
%
% Author: FOSSEE 
% Copyright (c) 2009, FOSSEE, IIT Bombay
%%%%%%%%%%%%%%%%%%%%%%%%%%%%%%%%%%%%%%%%%%%%%%%%%%%%%%%%%%%%%%%%%%%%%%%%%%%%%%%%

\documentclass[14pt,compress]{beamer}

\mode<presentation>
{
  \usetheme{Warsaw}
  \useoutertheme{infolines}
  \setbeamercovered{transparent}
}

\usepackage[english]{babel}
\usepackage[latin1]{inputenc}
%\usepackage{times}
\usepackage[T1]{fontenc}

% Taken from Fernando's slides.
\usepackage{ae,aecompl}
\usepackage{mathpazo,courier,euler}
\usepackage[scaled=.95]{helvet}

\definecolor{darkgreen}{rgb}{0,0.5,0}

\usepackage{listings}
\lstset{language=bash,
    basicstyle=\ttfamily\bfseries,
    commentstyle=\color{red}\itshape,
  stringstyle=\color{darkgreen},
  showstringspaces=false,
  keywordstyle=\color{blue}\bfseries}

\newcommand{\inctime}[1]{\addtocounter{time}{#1}{\tiny \thetime\ m}}

\newcommand{\typ}[1]{\lstinline{#1}}

\newcommand{\kwrd}[1]{ \texttt{\textbf{\color{blue}{#1}}}  }

\setbeamercolor{emphbar}{bg=blue!20, fg=black}
\newcommand{\emphbar}[1]


\begin{document}

\begin{frame}

\begin{center}
\vspace{12pt}
\textcolor{blue}{\huge {\LaTeX}: Typesetting Text}
\end{center}
\vspace{18pt}
\begin{center}
\vspace{10pt}
\includegraphics[scale=0.95]{../images/fossee-logo.png}\\
\vspace{5pt}
\scriptsize Developed by FOSSEE Team, IIT-Bombay. \\ 
\scriptsize Funded by National Mission on Education through ICT\\
\scriptsize  MHRD,Govt. of India\\
\includegraphics[scale=0.15]{../images/iitb-logo.jpg}\\
\end{center}
\end{frame}

\begin{frame}
  \frametitle{Objectives}
  At the end of this session, you will be able to:
  \begin{itemize}
  \item Learn how to typeset your document using {\LaTeX}
  \item Use lists, listings in your document for formatting text
  \end{itemize}
\end{frame}

\begin{frame}
  \frametitle{Prerequisite}
  Spoken Tutorial on:
  \begin{itemize}
  \item {\LaTeX} Part 1 - Getting Started
  \item {\LaTeX} Part 2 - Introduction
  \item {\LaTeX} Part 3 - Structuring the Content
  \end{itemize}
\end{frame}

\begin{frame}[fragile]
  \frametitle{Quotation Marks}
  \begin{itemize}
  \item Use \`~ (accent) for left quote
  \item Use \'~ (apostrophe) for right quote
  \item For double quotes, use them twice
  \end{itemize}
  \tiny See rev11 of \typ{hg}
\end{frame}

\begin{frame}[fragile]
  \frametitle{Fonts - Emphasis, Fixed width, \ldots}
  \begin{itemize}
  \item \lstinline{\emph} gives emphasized or italic text
  \item \typ{flushleft} to have text left aligned
  \item \typ{flushright}, \typ{center}
  \end{itemize}
  \tiny See rev12 of \typ{hg}
\end{frame}

\begin{frame}[fragile]
  \frametitle{Fonts - Emphasis, Fixed width, \ldots}
  \begin{itemize}
  \item \lstinline{\texttt} gives fixed width font
  \item \lstinline{\textbf} bold face font
  \item \lstinline{--} en dash (--); \lstinline{---} em dash (---). 
  \end{itemize}
  \tiny See rev13 of \typ{hg}
\end{frame}

\begin{frame}[fragile]
  \frametitle{Lists}
  \begin{itemize}
  \item \lstinline{enumerate} environment is used for numbered lists
  \item \lstinline{itemize} environment gives un-numbered lists
  \item Each item in the list is specified using \lstinline{\item}
  \item Nested lists are also easily handled, as expected
  \end{itemize}
  \tiny See rev14 of \typ{hg}
\end{frame}

\begin{frame}[fragile]
  \frametitle{Footnotes}
  \begin{itemize}
  \item \typ{\\footnote} command adds a footnote
  \end{itemize}
  \tiny See rev15 of \typ{hg}
\end{frame}

\begin{frame}[fragile]
  \frametitle{Labels and References}
  \begin{itemize}
  \item \lstinline+\label{labelname}+ is used to label an element
  \item \lstinline+\ref{labelname}+ is used to refer to that element
  \item Compile twice
  \end{itemize}
  \tiny See rev15 of \typ{hg}
\end{frame}

\begin{frame}[fragile]
  \frametitle{Including code}
  \begin{itemize}
  \item Instead of using \lstinline{\texttt} we could use
    \lstinline{\verbatim} 
  \item \lstinline+listings+ is a powerful package
  \item \lstinline+\usepackage{listings}+ needs to be added 
  \item Tell {\LaTeX} the language to be used, using \typ{\\lstset}
  \end{itemize}
  \tiny See rev16 of \typ{hg}
\end{frame}

\begin{frame}[fragile]
  \frametitle{Including code}
  \begin{itemize}
  \item Use \lstinline+\lstlisting+ for a block of code
  \item \typ{\\lstinline} for inline code
  \end{itemize}
  \tiny See rev16 of \typ{hg}
\end{frame}


\begin{frame}[fragile]
	\frametitle{Summary...}
	\begin{itemize}
	Put Quotation Marks around text
	\item How to Emphasize and give fixed width to fonts.
	\item Use numbered and un-numbered lists
	\item Add Footnotes, Labels and References
	\item Use the listings package to include code
	\end{itemize}
\end{frame}

\begin{frame}[fragile]
\frametitle{Evaluation}
\begin{enumerate}
\item 
\item 
\item 
\end{enumerate}
\end{frame}
\begin{frame}

\frametitle{Solutions}
\begin{enumerate}
\item 
\vspace{15pt}
\item 
\end{enumerate}
\end{frame}
\begin{frame}

\begin{block}{}
  \begin{center}
  \textcolor{blue}{\Large THANK YOU!} 
  \end{center}
  \end{block}
\begin{block}{}
  \begin{center}
    For more Information, visit our website\\
    \url{http://fossee.in/}
  \end{center}  
  \end{block}
\end{frame}

\end{document}
