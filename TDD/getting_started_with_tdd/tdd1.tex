\documentclass[12pt,presentation]{beamer}
\usepackage[utf8]{inputenc}
\usepackage[T1]{fontenc}
\usepackage{fixltx2e}
\usepackage{graphicx}
\usepackage{longtable}
\usepackage{float}
\usepackage{wrapfig}
\usepackage{soul}
\usepackage{textcomp}
\usepackage{marvosym}
\usepackage{wasysym}
\usepackage{latexsym}
\usepackage{amssymb}
\usepackage{hyperref}
\tolerance=1000
\usepackage[english]{babel} \usepackage{ae,aecompl}
\usepackage{mathpazo,courier,euler} \usepackage[scaled=.95]{helvet}
\usepackage{listings}
\lstset{language=Python, basicstyle=\ttfamily\bfseries,
commentstyle=\color{red}\itshape, stringstyle=\color{green},
showstringspaces=false, keywordstyle=\color{blue}\bfseries}
\providecommand{\alert}[1]{\textbf{#1}}

\title{SEES: Test Driven Development}
\author{FOSSEE}

\usetheme{Warsaw}\usecolortheme{default}\useoutertheme{infolines}\setbeamercovered{transparent}

%%%%%%%%%%%%%%%%%%%%%%%%%%%%%%%%%%%%%%%%%%%%%%%%%%%%%%%%%%%%%%%%%%%%%%%%%%%%%%%%%%%%%%%%%%%%%%%%%%%%%%%%%%%%%%%%%%%%%



\begin{document}

\begin{frame}
\begin{center}
\vspace{12pt}
\textcolor{blue}{\huge Test Driven Development \\Part I}
\end{center}
\vspace{18pt}
\begin{center}
\vspace{10pt}
\includegraphics[scale=0.95]{../images/fossee-logo.png}\\
\vspace{5pt}
\scriptsize Developed by FOSSEE Team, IIT-Bombay. \\ 
\scriptsize Funded by National Mission on Education through ICT\\
\scriptsize  MHRD,Govt. of India\\
\includegraphics[scale=0.30]{../images/iitb-logo.png}\\
\end{center}
\end{frame}
%%%%%%%%%%%%%%%%%%%%%%%%%%%%%%%%%%%%%%%%%%%%%%%%%%%%%%%%%%%%%%%%%%%
\section{Introduction}

\begin{frame}
  \frametitle{Objectives}
  At the end of this tutorial, you will be able to,
  \begin{itemize}
  \item Know what is TDD.
  \item Understand the use of test cases.
  \item Write simple tests for a function.
 
  \end{itemize}
\end{frame}
%%%%%%%%%%%%%%%%%%%%%%%%%%%%%%%%%%%%%%%%%%%%%%%%%%%%%%%%%%%%%%%%%%%%
\begin{frame}
\frametitle{Pre-requisite}
\label{sec-3}

Spoken tutorial on -
\begin{itemize}
\item Getting started with functions.
\end{itemize}
\end{frame}

%%%%%%%%%%%%%%%%%%%%%%%%%%%%%%%%%%%%%%%%%%%%%%%%%%%%%%%%%%%%%%%%%%%%%

\begin{frame}
  \frametitle{What is TDD?}
  The basic steps of TDD are roughly as follows --
  \begin{enumerate}
  \item Decide upon the feature to implement and the methodology of
    testing it.
  \item Write the tests for the feature decided upon.
  \item Just write enough code, so that the test can be run, but it fails.
  \item Improve the code, to just pass the test and at the same time
    passing all previous tests.
  \item Run the tests to see, that all of them run successfully.
  \item Refactor the code you've just written -- optimize the algorithm,
    remove duplication, add documentation, etc.
  \item Run the tests again, to see that all the tests still pass.
  
  \end{enumerate}
\end{frame}
%%%%%%%%%%%%%%%%%%%%%%%%%%%%%%%%%%%%%%%%%%%%%%%%%%%%%%%%%%%%%%%%%%%
\section{First Test}

\begin{frame}[fragile]
  \frametitle{First Test -- fibonacci}
  \begin{itemize}
  \item simple program -- Returns nth fibonacci number
  \item What are our code units?
    \begin{itemize}
    \item Only one function \texttt{fibonacci}
    \item Takes one number as argument
    \item Returns one number, which is the nth number of fibonacci series.
    \end{itemize}
\begin{lstlisting}
c = fibonacci(3)
\end{lstlisting}
  \item c will contain 3rd number of fibonacci series,counting from 0.
  \end{itemize}
\end{frame}
%%%%%%%%%%%%%%%%%%%%%%%%%%%%%%%%%%%%%%%%%%%%%%%%%%%%%%%%%%%%%%%%%%%%%%

\begin{frame}[fragile]
  \frametitle{Test Cases}
  \begin{itemize}
  \item Important to have test cases and expected outputs even before
    writing the first test!
  \item $n=3$, $fibonacci=2$
  \item $n=4$, $fibonacci=3$
  \item Tests are just a series of assertions
  \item True or False, depending on expected and actual behavior
  \end{itemize}
\end{frame}
%%%%%%%%%%%%%%%%%%%%%%%%%%%%%%%%%%%%%%%%%%%%%%%%%%%%%%%%%%%%%%%%%%%%%%

\begin{frame}[fragile]
  \frametitle{Test Cases -- Code}
\begin{lstlisting}
tc1 = fibonacci(3)
if tc1 != 2:
    print "Failed for n=3. Expected 2. \
    Obtained %d instead." % tc1
    exit(1)

tc2 = fibonacci(4)
if tc2 != 3:
    print "Failed for n=4. Expected 3. \
    Obtained %d instead." % tc2
    exit(1)

print "All tests passed!"
\end{lstlisting}
\begin{itemize}
\item The function \texttt{fibonacci} doesn't even exist!
\end{itemize}
\end{frame}
%%%%%%%%%%%%%%%%%%%%%%%%%%%%%%%%%%%%%%%%%%%%%%%%%%%%%%%%%%%%%%%%%%%%%%%%

\begin{frame}[fragile]
  \frametitle{Stubs}
  \begin{itemize}
  \item First write a very minimal definition of \texttt{fibonacci}
    \begin{lstlisting}
def fibonacci(a):
    pass
    \end{lstlisting}
  \item Written just, so that the tests can run
  \item Obviously, the tests are going to fail
  \end{itemize}
\end{frame}
%%%%%%%%%%%%%%%%%%%%%%%%%%%%%%%%%%%%%%%%%%%%%%%%%%%%%%%%%%%%%%%%%%%%%%%%%
\begin{frame}[fragile]
  \frametitle{\texttt{fibonacci.py}}
\begin{lstlisting}
def fibonacci(n):
    pass

if __name__ == '__main__':
    tc1 = fibonacci(3)
    if tc1 != 2:
        print "Failed for n=3. \
        Expected 2. Obtained %d instead." % tc1
        exit(1)
    tc2 = fibonacci(4)
    if tc2 != 3:
        print "Failed for n=4. \
        Expected 3. Obtained %d instead." % tc2
        exit(1)
    print "All tests passed!"
\end{lstlisting}
\end{frame}
%%%%%%%%%%%%%%%%%%%%%%%%%%%%%%%%%%%%%%%%%%%%%%%%%%%%%%%%%%%%%%%%%%%%%%%%%
\begin{frame}[fragile]
  \frametitle{First run}
\begin{lstlisting}
$ python fibonacci.py
Traceback (most recent call last):
  File "fibonacci.py", line 8, in <module> 
  print "Failed for n=3. Expected 2. 
  Obtained %d instead." % tc1
TypeError: %d format: 
a number is required, not NoneType
\end{lstlisting} %$

  \begin{itemize}
  \item We have our code unit stub, and a failing test. 
  \item The next step is to write code, so that the test just passes.
  \end{itemize}
\end{frame}
%%%%%%%%%%%%%%%%%%%%%%%%%%%%%%%%%%%%%%%%%%%%%%%%%%%%%%%%%%%%%%%%%%%%%%%%
\begin{frame}[fragile]
  \frametitle{Euclidean Algorithm}
  \begin{itemize}
  \item Modify the \texttt{fibonacci} stub function
  \item Then, run the script to see if the tests pass.
  \end{itemize}
\begin{lstlisting}
def fibonacci(n):
    a, b = 0, 1
    for i in range(n):
        a, b = b, a + b
    return a

\end{lstlisting}
\begin{lstlisting}
$ python fibonacci.py
All tests passed!
\end{lstlisting} %$
  \begin{itemize}
  \item \alert{Success!}
  \end{itemize}
\end{frame}
%%%%%%%%%%%%%%%%%%%%%%%%%%%%%%%%%%%%%%%%%%%%%%%%%%%%%%%%%%%%%%%%%%%%%%

\begin{frame}[fragile]
  \frametitle{Euclidean Algorithm -- Recursive}
  \begin{itemize}
  \item Final improvement -- make \texttt{fibonacci} recursive
  \item More readable and easier to understand
\begin{lstlisting}
def fibonacci(n):
    if n == 0:
        return 0
    elif n == 1:
        return 1
    else:
        return fibonacci(n-1) + fibonacci(n-2)
        
\end{lstlisting}
  \item Check that the tests pass again
  \end{itemize}
\end{frame}


%%%%%%%%%%%%%%%%%%%%%%%%%%%%%%%%%%%%%%%%%%%%%%%%%%%%%%%%%%%%%%%%%%%%%%%%%%%%%
\begin{frame}
\frametitle{Summary}
\label{sec-8}

  In this tutorial, we have learnt to,


\begin{itemize}
\item Undestand the basic steps involved in Test driven development.
\item Design a TDD approach for a given \texttt{fibonacci} function.

\end{itemize}
\end{frame}

%%%%%%%%%%%%%%%%%%%%%%%%%%%%%%%%%%%%%%%%%%%%%%%%%%%%%%%%%%%%%%%%%%%%%%%%%%%%%%%%%%%%%%
\begin{frame}[fragile]
\frametitle{Evaluation}
\label{sec-9}


\begin{enumerate}
\item ?
\vspace{8pt}
\item ? 
\end{enumerate}
\end{frame}
\begin{frame}
\frametitle{Solutions}
\label{sec-10}


\begin{enumerate}
\item 
\vspace{15pt}
\item 
\end{enumerate}
\end{frame}
%%%%%%%%%%%%%%%%%%%%%%%%%%%%%%%%%%%%%%%%%%%%%%%%%%%%%%%%%%%%%%%%%%%%%%%%%%%

\begin{frame}

  \begin{block}{}
  \begin{center}
  \textcolor{blue}{\Large THANK YOU!} 
  \end{center}
  \end{block}
\begin{block}{}
  \begin{center}
    For more Information, visit our website\\
    \url{http://fossee.in/}
  \end{center}  
  \end{block}
\end{frame}


\end{document}
