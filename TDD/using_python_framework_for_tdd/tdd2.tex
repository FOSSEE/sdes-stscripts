\documentclass[12pt,presentation]{beamer}
\usepackage[utf8]{inputenc}
\usepackage[T1]{fontenc}
\usepackage{fixltx2e}
\usepackage{graphicx}
\usepackage{longtable}
\usepackage{float}
\usepackage{wrapfig}
\usepackage{soul}
\usepackage{textcomp}
\usepackage{marvosym}
\usepackage{wasysym}
\usepackage{latexsym}
\usepackage{amssymb}
\usepackage{hyperref}
\tolerance=1000
\usepackage[english]{babel} \usepackage{ae,aecompl}
\usepackage{mathpazo,courier,euler} \usepackage[scaled=.95]{helvet}
\usepackage{listings}
\lstset{language=Python, basicstyle=\ttfamily\bfseries,
commentstyle=\color{red}\itshape, stringstyle=\color{green},
showstringspaces=false, keywordstyle=\color{blue}\bfseries}
\providecommand{\alert}[1]{\textbf{#1}}

\title{SEES: Test Driven Development}
\author{FOSSEE}

\usetheme{Warsaw}\usecolortheme{default}\useoutertheme{infolines}\setbeamercovered{transparent}

%%%%%%%%%%%%%%%%%%%%%%%%%%%%%%%%%%%%%%%%%%%%%%%%%%%%%%%%%%%%%%%%%%%%%%%%%%%%%%%%%%%%%%%%%%%%%%%%%%%%%%%%%%%%%%%%%%%%%



\begin{document}

\begin{frame}
\begin{center}
\vspace{12pt}
\textcolor{blue}{\huge Test Driven Development \\Part II}
\end{center}
\vspace{18pt}
\begin{center}
\vspace{10pt}
\includegraphics[scale=0.95]{../images/fossee-logo.png}\\
\vspace{5pt}
\scriptsize Developed by FOSSEE Team, IIT-Bombay. \\ 
\scriptsize Funded by National Mission on Education through ICT\\
\scriptsize  MHRD,Govt. of India\\
\includegraphics[scale=0.30]{../images/iitb-logo.png}\\
\end{center}
\end{frame}
%%%%%%%%%%%%%%%%%%%%%%%%%%%%%%%%%%%%%%%%%%%%%%%%%%%%%%%%%%%%%%%%%%%
\section{Introduction}

\begin{frame}
  \frametitle{Objectives}
  At the end of this tutorial, you will be able to,
  \begin{itemize}
  \item Know what are persistent test cases.
  \item Write doctest \& unittest for any given function.
  \item Understand the use of nosetest.
 
  \end{itemize}
  \end{frame}
%%%%%%%%%%%%%%%%%%%%%%%%%%%%%%%%%%%%%%%%%%%%%%%%%%%%%%%%%%%%%%%%%%%

\begin{frame}
\frametitle{Pre-requisite}
\label{sec-3}

Spoken tutorial on -
\begin{itemize}
\item Test Driven Development -- Part I
\end{itemize}
\end{frame}
%%%%%%%%%%%%%%%%%%%%%%%%%%%%%%%%%%%%%%%%%%%%%%%%%%%%%%%%%%%%%%%%%%%%

\begin{frame}[fragile]
  \frametitle{Persistent Test Cases}
  \begin{itemize}
  \item Tests should be pre-determined and written, before the code
  \item Test Data is repeatedly used; Hence, saved in persistent
    format
  \item Let's save data for fibonacci tests in a text file. 
  \item The file shall have multiple lines of test data
  \item Each line corresponds to a single test case
  \item Each line consists of two comma separated values --
    \begin{itemize}
    \item First coloumn is the integer which has to be
      passed to the function.
    \item Second coloumn is the return value from the function.
    \end{itemize}
  \item Let us call our data file \texttt{fibonacci\_testcases.dat}
  \end{itemize}
\end{frame}
%%%%%%%%%%%%%%%%%%%%%%%%%%%%%%%%%%%%%%%%%%%%%%%%%%%%%%%%%%%%%%%%%%%
\begin{frame}[fragile]
  \frametitle{Modify \texttt{fibonacci.py}}
\begin{lstlisting}
if __name__ == '__main__':
    for line in open('fibonacci_testcases.dat'):
        values = line.split(', ')
        n = int(values[0])
        a = int(values[1])

        tc = fibonacci(n)
        if tc != a:
            print "Failed for n=%d.\
            Expected %d. Obtained %d instead."\
            % (n, a, tc)
            exit(1)

    print "All tests passed!"
\end{lstlisting}
\end{frame}
%%%%%%%%%%%%%%%%%%%%%%%%%%%%%%%%%%%%%%%%%%%%%%%%%%%%%%%%%%%%%%%%%%
\section{Python Testing Frameworks}

\begin{frame}[fragile]
  \frametitle{Python Testing Frameworks}
  \begin{itemize}
  \item Testing frameworks essentially, ease the job of the user
  \item Python provides two frameworks for testing code
    \begin{itemize}
    \item \texttt{unittest} framework
    \item \texttt{doctest} module
    \end{itemize}
  \end{itemize}
\end{frame}
%%%%%%%%%%%%%%%%%%%%%%%%%%%%%%%%%%%%%%%%%%%%%%%%%%%%%%%%%%%%%%%%%
\subsection{\texttt{doctest} module}

\begin{frame}[fragile]
  \frametitle{doctest}
  \begin{itemize}
  \item Documentation always accompanies a well written piece of code
  \item Use \texttt{docstring} to document functions or modules
  \item Along with description and usage, examples can be added
  \item Interactive interpreter session inputs and outputs are
    copy-pasted 
  \item \texttt{doctest} module picks up all such interactive examples
  \item Executes them and determines if the code runs, as documented
  \end{itemize}
  Let's use the \texttt{doctest} module for our \texttt{fibonacci} function
\end{frame}

%%%%%%%%%%%%%%%%%%%%%%%%%%%%%%%%%%%%%%%%%%%%%%%%%%%%%%%%%%%%%%%%%

\begin{frame}[fragile]
  \frametitle{doctest for \texttt{fibonacci.py}}
\begin{tiny}
\begin{lstlisting}
def fibonacci(n):
    """Returns the nth fibonacci number.

    Args:
      n: an integer
    
    
    >>> fibonacci(3)
    2
    >>> fibonacci(4)
    3
    """
    if n == 0:
        return 0
    elif n == 1:
        return 1
    else:
        return fibonacci(n-1) + fibonacci(n-2) 
\end{lstlisting}
\end{tiny}
\begin{itemize}
\item We have added examples to the \texttt{docstring}
\item Now we need to tell the \texttt{doctest} module to execute
\end{itemize}
\end{frame}
%%%%%%%%%%%%%%%%%%%%%%%%%%%%%%%%%%%%%%%%%%%%%%%%%%%%%%%%%%%%%%%%%%%%%%%
\begin{frame}[fragile]
  \frametitle{doctest for \texttt{fibonacci.py} \ldots}
\begin{lstlisting}
if __name__ == "__main__":
    import doctest
    doctest.testmod()
\end{lstlisting}
\begin{itemize}
\item \texttt{testmod} automatically picks all sample sessions
\item Executes them and compares output with documented output
\item It doesn't give any output, when all the results match 
\item Complains only when one or more tests fail. 
\end{itemize}
\end{frame}
%%%%%%%%%%%%%%%%%%%%%%%%%%%%%%%%%%%%%%%%%%%%%%%%%%%%%%%%%%%%%%%%%%%%%%%%%%%
\begin{frame}[fragile]
  \frametitle{\texttt{doctest} -- Execution}
  \begin{itemize}
  \item Run the doctests by running \texttt{fibonacci.py}
\begin{lstlisting}
$ python fibonacci.py
\end{lstlisting} %$
  \item All the tests pass, and doctest gives no output
  \item For a more detailed report we can run with -v argument
\begin{lstlisting}
$ python fibonacci.py -v
\end{lstlisting} %$
  \item If the output contains a blank line, use \texttt{<BLANKLINE>}
  \item To see a failing test case, replace \texttt{return a} with \texttt{b}
  \end{itemize}
\end{frame}

\subsection{\texttt{unittest} framework}
%%%%%%%%%%%%%%%%%%%%%%%%%%%%%%%%%%%%%%%%%%%%%%%%%%%%%%%%%%%%%%%%%%%%%%%%%%
\begin{frame}[fragile]
  \frametitle{\texttt{unittest}}
  \begin{itemize}
  \item It won't be long, before we complain about the power of
    \texttt{doctest} 
  \item \texttt{unittest} framework can efficiently automate tests
  \item Easily initialize code and data for executing the specific
    tests
  \item Cleanly shut them down once the tests are executed
  \item Easily aggregate tests into collections and improved reporting
  \end{itemize}
\end{frame}
%%%%%%%%%%%%%%%%%%%%%%%%%%%%%%%%%%%%%%%%%%%%%%%%%%%%%%%%%%%%%%%%%%%%%%

\begin{frame}[fragile]
  \frametitle{\texttt{unittest}ing \texttt{fibonacci.py}}
  \begin{itemize}
  \item Subclass the \texttt{TestCase} class in \texttt{unittest}
  \item Place all the test code as methods of this class
  \item Use the test cases present in \texttt{fibonacci\_testcases.dat}
  \item Place the code in \texttt{test\_fibonacci.py} 
  \end{itemize}
\end{frame}
%%%%%%%%%%%%%%%%%%%%%%%%%%%%%%%%%%%%%%%%%%%%%%%%%%%%%%%%%%%%%%%%%%%%%%%%%%%%%%%%%
\begin{frame}[fragile,allowframebreaks]
  \frametitle{\texttt{test\_fibonacci.py}}
\small
\begin{lstlisting}
import fibonacci
import unittest

class TestFibonacciFunction(unittest.TestCase):

    def setUp(self):
        self.test_file = \
        open('fibonacci_testcases.dat')
        self.test_cases = []
        for line in self.test_file:
            values = line.split(', ')
            n = int(values[0])
            a = int(values[1])
            
            self.test_cases.append([n, a])

    def test_fibonacci(self):
        for case in self.test_cases:
            n = case[0]
            a = case[1]
            self.assertEqual(fibonacci.fibonacci(n),a)

    def tearDown(self):
        self.test_file.close()
        del self.test_cases

if __name__ == '__main__':
    unittest.main()
\end{lstlisting}
\end{frame}
%%%%%%%%%%%%%%%%%%%%%%%%%%%%%%%%%%%%%%%%%%%%%%%%%%%%%%%%%%%%%%%%%%%%%%%%
\begin{frame}[fragile]
  \frametitle{\texttt{test\_fibonacci.py}}
  \begin{itemize}
  \item \texttt{setUp} -- we read all the test data and store it in a
    list
  \item \texttt{tearDown} -- delete the data to free up memory and
    close open file
  \item \texttt{test\_fibonacci} -- actual test code
  \item \texttt{assertEqual} -- compare actual result with expected one
  \item Write documentation for above code. 
  \end{itemize}
\end{frame}

\section{\texttt{nose}}
%%%%%%%%%%%%%%%%%%%%%%%%%%%%%%%%%%%%%%%%%%%%%%%%%%%%%%%%%%%%%%%%%%%%%%%%%%%
\begin{frame}[fragile]
  \frametitle{\texttt{nose} tests}
  \begin{itemize}
  \item It is not easy to organize, choose and run tests scattered
    across multiple files. 
  \item \texttt{nose} module aggregate these tests automatically
  \item Can aggregate \texttt{unittests} and \texttt{doctests}
  \item Allows us to pick and choose which tests to run
  \item Helps output the test-results and aggregate them in various
    formats
  \item Not part of the Python distribution itself
\begin{lstlisting}
$ easy_install nose
\end{lstlisting} %$
  \item Run the following command in the top level directory
\begin{lstlisting}
$ nosetests
\end{lstlisting} %$
  \end{itemize}
\end{frame}

%%%%%%%%%%%%%%%%%%%%%%%%%%%%%%%%%%%%%%%%%%%%%%%%%%%%%%%%%%%%%%%%%%%%%%%%%%%%%
\begin{frame}
\frametitle{Summary}
\label{sec-8}

  In this tutorial, we have learnt to,


\begin{itemize}
\item Use of persistent test cases for better control.
\item Undestand the use of doctest \& unittest.
\item Understand the use of nosetest .

\end{itemize}
\end{frame}

%%%%%%%%%%%%%%%%%%%%%%%%%%%%%%%%%%%%%%%%%%%%%%%%%%%%%%%%%%%%%%%%%%%%%%%%%%%%%%%%%%%%%%
\begin{frame}[fragile]
\frametitle{Evaluation}
\label{sec-9}


\begin{enumerate}
\item An underlying assumption of TDD is that you have a testing framework available to you.\\
  \begin{enumerate}
  \item True
  \item False
  \end{enumerate}
\vspace{8pt}
\item Test-Driven Development is an advanced technique of using automated \_\_\_\_\_\_\_\_\_ tests to drive the design of software and force deoupling of dependecies.\\
  \begin{enumerate}
  \item Incremental
  \item Unit
  \item Programming
  \item Object
  \end{enumerate}
\end{enumerate}
\end{frame}
\begin{frame}
\frametitle{Solutions}
\label{sec-10}


\begin{enumerate}
\item True
\vspace{15pt}
\item Unit
\end{enumerate}
\end{frame}
%%%%%%%%%%%%%%%%%%%%%%%%%%%%%%%%%%%%%%%%%%%%%%%%%%%%%%%%%%%%%%%%%%%%%%%%%%%

\begin{frame}

  \begin{block}{}
  \begin{center}
  \textcolor{blue}{\Large THANK YOU!} 
  \end{center}
  \end{block}
\begin{block}{}
  \begin{center}
    For more Information, visit our website\\
    \url{http://fossee.in/}
  \end{center}  
  \end{block}
\end{frame}


\end{document}
