%%%%%%%%%%%%%%%%%%%%%%%%%%%%%%%%%%%%%%%%%%%%%%%%%%%%%%%%%%%%%%%%%%%%%%%%%%%%%%%%
% Using Linux Tools
%
% Author: FOSSEE 
% Copyright (c) 2009, FOSSEE, IIT Bombay
%%%%%%%%%%%%%%%%%%%%%%%%%%%%%%%%%%%%%%%%%%%%%%%%%%%%%%%%%%%%%%%%%%%%%%%%%%%%%%%%

\documentclass[17pt,compress]{beamer}
\usepackage{beamerthemesplit}
\mode<presentation>
{
  \usetheme{Warsaw}
  \useoutertheme{infolines}
  \setbeamercovered{transparent}
  \setbeamertemplate{navigation symbols}{}
}
% Taken from Fernando's slides.
\usepackage{ae,aecompl}
\usepackage[scaled=.95]{helvet}

\usepackage[english]{babel}
\usepackage[latin1]{inputenc}
\usepackage[T1]{fontenc}

% change the alerted colour to LimeGreen
\definecolor{LimeGreen}{RGB}{50,205,50}
\setbeamercolor{structure}{fg=LimeGreen}
\author[FOSSEE]{}
\institute[IIT Bombay]{}
\date[]{}
% \setbeamercovered{transparent}

% theme split
\usepackage{verbatim}
\newenvironment{colorverbatim}[1][]%
{%
\color{blue}
\verbatim
}%
{%
\endverbatim
}%

\usepackage{mathpazo,courier,euler}
\usepackage{listings}
\lstset{language=sh,
    basicstyle=\ttfamily\bfseries,
  showstringspaces=false,
  keywordstyle=\color{black}\bfseries}

% logo
\logo{\includegraphics[height=1.30 cm]{../images/3t-logo.pdf}}
\logo{\includegraphics[height=1.30 cm]{../images/fossee-logo.pdf}

\hspace{7.5cm}
\includegraphics[scale=0.99]{../images/fossee-logo.pdf}\\
\hspace{281pt}
\includegraphics[scale=0.80]{../images/3t-logo.pdf}}
%%%%%%%%%%%%%%%%%%%%%%%%%%%%%%%%%%%%%%%%%%%%%%%%%%%%%%%%%%%%%%%%%%%%%%
% DOCUMENT STARTS
\begin{document}

\sffamily \bfseries
\title
[Miscellaneous Tools]
{Miscellaneous Tools}
\author
[FOSSEE]
{\small Talk to a Teacher\\{\color{blue}\url{http://spoken-tutorial.org}}\\\vspace{0.25cm}National Mission on Education
 through ICT\\{\color{blue}\url{ http://sakshat.ac.in}} \\ [1.9cm]
   Contributed by FOSSEE Team \\IIT Bombay  \\[0.3cm]
}

% slide 1
\begin{frame}
   \titlepage
\end{frame}


\begin{frame}
\frametitle{Objectives}
\label{sec-2}

At the end of this tutorial, you will be able to,
\begin{itemize}
\item Search for files in various ways
\item Compare files with same names
\item Create and extract an archive
\item Customize a shell
\end{itemize}
\end{frame}

\begin{frame}
\frametitle{Pre-requisites}
\label{sec-3}

Spoken tutorial on,
\begin{itemize}
\item Getting started with Linux
\item Basic File Handling
\end{itemize}
\end{frame}

\begin{frame}[fragile]
  \frametitle{\texttt{`find'}}
  \begin{itemize}
  \item `find' command helps to find files in a directory hierarchy
  \item Offers a very complex feature set\\ For eg: search files by name, owner, date,etc.
  \item Look at the \texttt{man} page of `find' 
  \end{itemize}
\end{frame}

\begin{frame}[fragile]
  \frametitle{\texttt{`cmp'}}
  \begin{lstlisting}
   $ find . -name quick.c
   ./Desktop/programs/quick.c
   ./c-folder/quick.c
   $ cmp Desktop/programs/quick.c 
     c-folder/quick.c
  \end{lstlisting} % $
  \begin{itemize}
  \item No output when files are exactly same
  \item Else, gives location where the first difference occurs 
  \end{itemize}
\end{frame}

\begin{frame}[fragile]
  \frametitle{\texttt{`diff'}}
  \begin{itemize}
  \item We know the files are different, but want exact differences i.e. line by line
  \end{itemize}
\hspace{30pt}\verb~$diff Desktop/programs/quick.c ~\\
\hspace{40pt}\verb~c-folder/quick.c~
 
  \begin{itemize}
  \item \texttt{>} indicates content only in second file
  \item \texttt{<} indicates content only in first file
  \end{itemize}
\end{frame}

\begin{frame}[fragile]
\frametitle{\texttt{`tar'}}
\begin{itemize}
\item \emph{tarball} -- essentially a collection of files
\item May or may not be compressed
\item Eases the job of storing, backing-up \& transporting files
\end{itemize}
\end{frame}

\begin{frame}[fragile]
\frametitle{Extracting an archive}
\verb~mkdir extract~
\verb~cp allfiles.tar extract/~
\verb~cd extract~\\
\verb~tar -xvf allfiles.tar~


\begin{itemize}
\item \texttt{-x} --- Extract files within the archive
\item \texttt{-f} --- Specify the archive file
\item \texttt{-v} --- Be verbose
\end{itemize}
\end{frame}

\begin{frame}[fragile]
  \frametitle{Compressed archives}
  \begin{itemize}
  \item \texttt{tar} - create \& extract compressed archives
  \item Additional option to handle compressed archives
    \begin{center}
      \begin{tabular}{|l|l|}\hline
        Compression      &  Option   \\\hline
        gzip   &  \texttt{-z}        \\\hline
        bzip2  &  \texttt{-j}        \\\hline
        lzma   &  \texttt{-{}-lzma}  \\\hline
      \end{tabular}
    \end{center}
  \end{itemize}
\end{frame}


\begin{frame}
\frametitle{Customizing your shell}
\begin{itemize}
\item Bash reads \texttt{/etc/profile},
  \texttt{\textasciitilde{}/.bash\_profile},
  \texttt{\textasciitilde{}/.bash\_login}, and
  \texttt{\textasciitilde{}/.profile} in that order, when starting
  up as a login shell. 
\item \texttt{\textasciitilde{}/.bashrc} is read, when not a login
  shell 
\end{itemize}
\end{frame}


\begin{frame}
\frametitle{Summary}
\label{sec-8}

  In this tutorial, we have learnt to,


\begin{itemize}
\item Use ``find'' command to find files in a directory hierarchy
\item Find differences between files with the same name, using the
    ``cmp'' and ``diff'' commands
\end{itemize}
\end{frame}

\begin{frame}
\frametitle{Summary..}
\label{sec-8}

\begin{itemize}
\item Extract and create compressed archive's using the ``tar'' command
\item Customize one's shell according to one's choice
\end{itemize}
\end{frame}

\begin{frame}[fragile]
\frametitle{Evaluation}
\label{sec-9}


\begin{enumerate}
\item Look at the man page of ``find'' and state the options which
    deal with symbolic links.
\vspace{8pt}
\item How do you append tar files to an archive ?
\end{enumerate}
\end{frame}
\begin{frame}
\frametitle{Solutions}

\begin{enumerate}
\item  -H,  -L  and  -P   
\vspace{15pt}
\item \$tar -Af <tar-file><tar-file-to-be-added>
\begin{center}
OR \\
\end{center}
\$tar -rf <tar-file><tar-file-to-be-added>
\end{enumerate}

\end{frame}
\begin{frame}

  \begin{block}{}
  \begin{center}
  {\Large THANK YOU!} 
  \end{center}
  \end{block}
\begin{block}{}
  \begin{center}
    For more Information, visit our website\\
    {\color{blue}\url{http://fossee.in/}}
  \end{center}  
  \end{block}
\end{frame}

\end{document}



