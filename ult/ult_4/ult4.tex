%%%%%%%%%%%%%%%%%%%%%%%%%%%%%%%%%%%%%%%%%%%%%%%%%%%%%%%%%%%%%%%%%%%%%%%%%%%%%%%%
% Using Linux Tools
%
% Author: FOSSEE 
% Copyright (c) 2009, FOSSEE, IIT Bombay
%%%%%%%%%%%%%%%%%%%%%%%%%%%%%%%%%%%%%%%%%%%%%%%%%%%%%%%%%%%%%%%%%%%%%%%%%%%%%%%%


\documentclass[17pt,compress]{beamer}
\usepackage{beamerthemesplit}
\mode<presentation>
{
  \usetheme{Warsaw}
  \useoutertheme{infolines}
  \setbeamercovered{transparent}
  \setbeamertemplate{navigation symbols}{}
}
% Taken from Fernando's slides.
\usepackage{ae,aecompl}
\usepackage[scaled=.95]{helvet}

\usepackage[english]{babel}
\usepackage[latin1]{inputenc}
\usepackage[T1]{fontenc}

% change the alerted colour to LimeGreen
\definecolor{LimeGreen}{RGB}{50,205,50}
\setbeamercolor{structure}{fg=LimeGreen}
\author[FOSSEE]{}
\institute[IIT Bombay]{}
\date[]{}
% \setbeamercovered{transparent}

% theme split
\usepackage{verbatim}
\newenvironment{colorverbatim}[1][]%
{%
\color{blue}
\verbatim
}%
{%
\endverbatim
}%

\usepackage{mathpazo,courier,euler}
\usepackage{listings}
\lstset{language=sh,
    basicstyle=\ttfamily\bfseries,
  showstringspaces=false,
  keywordstyle=\color{black}\bfseries}

% logo
\logo{\includegraphics[height=1.30 cm]{3t-logo.pdf}}
\logo{\includegraphics[height=1.30 cm]{fossee-logo.pdf}

\hspace{7.5cm}
\includegraphics[scale=0.99]{fossee-logo.pdf}\\
\hspace{281pt}
\includegraphics[scale=0.80]{3t-logo.pdf}}

%%%%%%%%%%%%%%%%%%%%%%%%%%%%%%%%%%%%%%%%%%%%%%%%%%%%%%%%%%%%%%%%%%%%%%
% DOCUMENT STARTS
\begin{document}

\sffamily \bfseries
\title
[Advanced File Handling]
{Advanced File Handling}
\author
[FOSSEE]
{\small Talk to a Teacher\\{\color{blue}\url{http://spoken-tutorial.org}}\\\vspace{0.25cm}National Mission on Education
 through ICT\\{\color{blue}\url{ http://sakshat.ac.in}} \\ [1.65cm]
   Contributed by FOSSEE Team \\IIT Bombay  \\[0.3cm]
}

% slide 1
\begin{frame}
   \titlepage
\end{frame}

\begin{frame}
\frametitle{Objectives}
\label{sec-2}

At the end of this tutorial, you will be able to,
\begin{itemize}
\item Display the contents of files
\item Read only parts of a file
\item Look at the statistical information of a file
\end{itemize}
\end{frame}

\begin{frame}
\frametitle{Pre-requisites}
\label{sec-3}

Spoken tutorial on,
\begin{itemize}
\item Getting started with Linux
\item Basic File Handling
\end{itemize}
\end{frame}

\begin{frame}[fragile]
  \frametitle{\texttt{less}}
   \begin{itemize}
  \item q: Quit
  \item Arrows/Page Up/Page Down/Home/End: Navigation
  \item ng: Jump to line number n
  \item /pattern: Search. Regular expressions can be used
  \item h: Help
  \end{itemize}
\end{frame}

\begin{frame}
  \frametitle{Exercise 1}
  \begin{itemize}
  \item Print only the first, fifth and the seventh fields of the file \verb~/etc/passwd~
  \end{itemize}
\end{frame}

\begin{frame}[fragile]
  \frametitle{\texttt{paste}}
      \begin{center}
      \begin{tabular}{l|l}
        \verb~students.txt~  &  \verb~marks.txt~  \\
        Hussain              &  89 92 85          \\
        Dilbert              &  98 47 67          \\
        Anne                 &  67 82 76          \\
        Raul                 &  78 97 60          \\
        Sven                 &  67 68 69          \\
      \end{tabular}
    \end{center}
\end{frame}

\begin{frame}
\frametitle{Summary}
\label{sec-8}

  In this tutorial, we have learnt to,

\begin{itemize}
\item Display the contents of files using the ``cat'' command.
\item View the contents of a file one screen at a time using the 
      ``less'' command.
\end{itemize}
\end{frame}

\begin{frame}
\frametitle{Summary...}
\begin{itemize}
\item Display specific contents of file using the ``head'' and 
      ``tail'' commands.
\item Use the ``cut'', ``paste'' and ``wc'' commands.
\end{itemize}
\end{frame}

\begin{frame}[fragile]
\frametitle{Evaluation}
\label{sec-9}


\begin{enumerate}
\item How to view lines from 1 to 15 in \verb~wonderland.txt~ ?
\vspace{15pt}
\item In ``cut'' command, how to specify space as the delimiter ? 
\end{enumerate}
\end{frame}
\begin{frame}
\frametitle{Solutions}
\label{sec-10}


\begin{enumerate}
\item \$ head -15 \verb~wonderland.txt~
\vspace{15pt}
\item \$ cut -d " " <filename>
\end{enumerate}
\end{frame}

\begin{frame}
\frametitle{SDES \& FOSSEE}
\begin{center}
\begin{itemize}
\item \small{SDES}\\
\small{\color{LimeGreen}Software Development techniques for Engineers and Scientists} \\
\scriptsize An initiative by FOSSEE. \\
\vspace{3pt}
\scriptsize For more information on SDES, please visit {\color{blue}\url{http://fossee.in/sdes}}\\
\vspace{10pt}
\item \small{FOSSEE}\\
\small {\color{LimeGreen}Free and Open-source Software for \\Science and Engineering Education} \\
\scriptsize Based at IIT Bombay, Funded by MHRD.\\
\vspace{3pt}
\scriptsize Part of National Mission on Education through ICT \\(NME-ICT) \\
\end{itemize}
\end{center}
\end{frame}

\begin{frame}
\frametitle{About the Spoken Tutorial Project}
\begin{itemize}
\item Watch the video available at {\color{blue}\url{http://spoken-tutorial.org /What\_is\_a\_Spoken\_Tutorial}} 
\item It summarises the Spoken Tutorial project 
\item If you do not have good bandwidth, you can download and watch it
\end{itemize}
\end{frame}

\begin{frame}
\frametitle{Spoken Tutorial Workshops}The Spoken Tutorial Project Team 
\begin{itemize}
\item Conducts workshops using spoken tutorials 
\item Gives certificates to those who pass an online test 
\item For more details, please write to \\ \hspace {0.5cm}{\color{blue}contact@spoken-tutorial.org}
\end{itemize}
\end{frame}

\begin{frame}
\frametitle{Acknowledgements}
\begin{itemize}
\item Spoken Tutorial Project is a part of the Talk to a Teacher  project 
\item It is supported by the National Mission on Education through  ICT, MHRD, Government of India 
\item More information on this Mission is available at: \\{\color{blue}\url{http://spoken-tutorial.org/NMEICT-Intro}}
\end{itemize}
\end{frame}

\begin{frame}

  \begin{block}{}
  \begin{center}
  {\Large THANK YOU!} 
  \end{center}
  \end{block}
\begin{block}{}
  \begin{center}
    For more Information, visit our website\\
    {\color{blue}\url{http://fossee.in/}}
  \end{center}  
  \end{block}
\end{frame}

\end{document}


