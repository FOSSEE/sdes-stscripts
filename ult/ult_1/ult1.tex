\documentclass[17pt,compress]{beamer}
\usepackage{beamerthemesplit}
\mode<presentation>
{
  \usetheme{Warsaw}
  \useoutertheme{infolines}
  \setbeamercovered{transparent}
  \setbeamertemplate{navigation symbols}{}
}
% Taken from Fernando's slides.
\usepackage{ae,aecompl}
\usepackage[scaled=.95]{helvet}

\usepackage[english]{babel}
\usepackage[latin1]{inputenc}
\usepackage[T1]{fontenc}

% change the alerted colour to LimeGreen
\definecolor{LimeGreen}{RGB}{50,205,50}
\setbeamercolor{structure}{fg=LimeGreen}
\author[FOSSEE]{}
\institute[IIT Bombay]{}
\date[]{}
% \setbeamercovered{transparent}

% theme split
\usepackage{verbatim}
\newenvironment{colorverbatim}[1][]%
{%
\color{blue}
\verbatim
}%
{%
\endverbatim
}%

\usepackage{mathpazo,courier,euler}
\usepackage{listings}
\lstset{language=sh,
    basicstyle=\ttfamily\bfseries,
  showstringspaces=false,
  keywordstyle=\color{black}\bfseries}

% logo
\logo{\includegraphics[height=1.30 cm]{3t-logo.pdf}}
\logo{\includegraphics[height=1.30 cm]{fossee-logo.pdf}

\hspace{7.5cm}
\includegraphics[scale=0.99]{../images/fossee-logo.pdf}\\
\hspace{281pt}
\includegraphics[scale=0.80]{../images/3t-logo.pdf}}

\begin{document}

\sffamily \bfseries
\title
[Getting started with Linux]
{Getting started with Linux}
\author
[FOSSEE]
{\small Talk to a Teacher\\{\color{blue}\url{http://spoken-tutorial.org}}\\\vspace{0.25cm}National Mission on Education
 through ICT\\{\color{blue}\url{ http://sakshat.ac.in}} \\ [1.65cm]
   Contributed by FOSSEE Team \\IIT Bombay  \\[0.3cm]
}

% slide 1
\begin{frame}
   \titlepage
\end{frame}

\begin{frame}
\frametitle{Objectives}
\label{sec-2}

At the end of this tutorial, you will be able to,
\begin{itemize}
\item Know what is Linux
\item Understand the need for Linux in today's world
\item Move around in directories and files
\item Use basic commands of Linux
\end{itemize}
\end{frame}

\begin{frame}[fragile]
  \begin{block}{What is \textbf{GNU/Linux} Operating System?}
    \begin{itemize}
    \item Free Open Source Operating System\\
      {\color{LimeGreen}{Free}} Free as in Free Speech\\
      {\color{LimeGreen}{Open-Source}} permits modifications \& redistribution of source code
      \item Unix-inspired \& runs on a variety of hardware
    \item Linux Kernel + Application software
    \end{itemize}
  \end{block}
\end{frame}

\begin{frame}[fragile]
  \frametitle{Why Linux?}
    \begin{itemize}
    \item Free, secure \& versatile
    \end{itemize}

    \begin{block}{Why Linux for Scientific Computing?}
      \begin{itemize}
        \item Can run forever
        \item Libraries
        \item Parallel Computing
      \end{itemize}
    \end{block}
\end{frame}

\begin{frame}[fragile]
  \frametitle{Logging in}
  \begin{itemize}
  \item GNU/Linux does have a GUI
  \item Command Line for this module
  \item Hit \texttt{Ctrl + Alt + F1}
  \item \texttt{logout} command logs you out
  \item Hit \texttt{Ctrl + Alt + F7} to come back to GUI
  \end{itemize}
\end{frame}

\begin{frame}[fragile]
  \frametitle{New folders}
  \begin{itemize}
  \item Special characters need to be escaped OR quoted
  \end{itemize}
  \begin{lstlisting}
    $ mkdir software\ engineering
    $ mkdir "software engg"
  \end{lstlisting} 
  \begin{itemize}
  \item Generally, use hyphens or underscores instead of spaces in names
  \end{itemize}
\end{frame}

\begin{frame}[fragile]
  \frametitle{Using additional options}

  \begin{itemize}
  \item \texttt{-h} or \texttt{---help} gives summary of command usage
  \end{itemize}
  \begin{lstlisting}
    $ ls -h
    $ ls --help
  \end{lstlisting} % $
\end{frame}

\begin{frame}
  \frametitle{Exercise 1}
  Which option should be used with 'ls' command to list all\\
  \begin{itemize}
  	\item the directories,\\
        \item the sub-directories,\\
	\item the files contained in it
  \end{itemize}
        Hint: Use \texttt{man} or \texttt{---help}
\end{frame}

\begin{frame}
\frametitle{Summary}
\label{sec-8}

  In this tutorial, we have learnt to,


\begin{itemize}
\item Understand the basic structure of Linux and it's need
\item Use ``pwd'' command, to check the current working directory
\item List a directory's contents by using the command ``ls''

\end{itemize}
\end{frame}

\begin{frame}
\frametitle{Summary...}
\label{sec-9}
\begin{itemize}
\item Use ``mkdir'' to create, ``rmdir'' to delete directories
\item Use ``man'' and ``whatis'' to get a description of a command
\item Use ``apropos'' to search manual page names and descriptions
\end{itemize}
\end{frame}

\begin{frame}[fragile]
\frametitle{Evaluation}
\label{sec-10}


\begin{enumerate}
\item What is the default directory after logging into the terminal?
\vspace{8pt}
\item How to view file attributes with ``ls'' command? 
\end{enumerate}
\end{frame}
\begin{frame}
\frametitle{Solutions}
\label{sec-11}


\begin{enumerate}
\item /home/user
\vspace{15pt}
\item ls -l <filename>
\end{enumerate}
\end{frame}

\begin{frame}
\frametitle{SDES \& FOSSEE}
\begin{center}
\begin{itemize}
\item \small{SDES}\\
\small{\color{LimeGreen}Software Development techniques for Engineers and Scientists} \\
\scriptsize An initiative by FOSSEE. \\
\vspace{3pt}
\scriptsize For more information on SDES, please visit {\color{blue}\url{http://fossee.in/sdes}}\\
\vspace{12pt}
\item \small{FOSSEE}\\
\small {\color{LimeGreen}Free and Open-source Software for \\Science and Engineering Education} \\
\scriptsize Based at IIT Bombay, Funded by MHRD.\\
\vspace{3pt}
\scriptsize Part of National Mission on Education through ICT (NME-ICT). \\
\end{itemize}
\end{center}
\end{frame}

\begin{frame}
\frametitle{About the Spoken Tutorial Project}
\begin{itemize}
\item Watch the video available at {\color{blue}\url{http://spoken-tutorial.org /What\_is\_a\_Spoken\_Tutorial}} 
\item It summarises the Spoken Tutorial project 
\item If you do not have good bandwidth, you can download and watch it
\end{itemize}
\end{frame}

\begin{frame}
\frametitle{Spoken Tutorial Workshops}The Spoken Tutorial Project Team 
\begin{itemize}
\item Conducts workshops using spoken tutorials 
\item Gives certificates to those who pass an online test 
\item For more details, please write to \\ \hspace {0.5cm}{\color{blue}contact@spoken-tutorial.org}
\end{itemize}
\end{frame}

\begin{frame}
\frametitle{Acknowledgements}
\begin{itemize}
\item Spoken Tutorial Project is a part of the Talk to a Teacher  project 
\item It is supported by the National Mission on Education through  ICT, MHRD, Government of India 
\item More information on this Mission is available at: \\{\color{blue}\url{http://spoken-tutorial.org/NMEICT-Intro}}
\end{itemize}
\end{frame}

\begin{frame}
  \begin{block}{}
  \begin{center}
  {\Large THANK YOU!} 
  \end{center}
  \end{block}
\begin{block}{}
  \begin{center}
    For more Information, visit our website\\
    {\color{blue}\url{http://fossee.in/}}
  \end{center}  
  \end{block}
\end{frame}

\end{document}
