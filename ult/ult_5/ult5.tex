%%%%%%%%%%%%%%%%%%%%%%%%%%%%%%%%%%%%%%%%%%%%%%%%%%%%%%%%%%%%%%%%%%%%%%%%%%%%%%%%
% Using Linux Tools
%
% Author: FOSSEE 
% Copyright (c) 2009, FOSSEE, IIT Bombay
%%%%%%%%%%%%%%%%%%%%%%%%%%%%%%%%%%%%%%%%%%%%%%%%%%%%%%%%%%%%%%%%%%%%%%%%%%%%%%%%

\documentclass[17pt,compress]{beamer}
\usepackage{beamerthemesplit}
\mode<presentation>
{
  \usetheme{Warsaw}
  \useoutertheme{infolines}
  \setbeamercovered{transparent}
  \setbeamertemplate{navigation symbols}{}
}
% Taken from Fernando's slides.
\usepackage{ae,aecompl}
\usepackage[scaled=.95]{helvet}

\usepackage[english]{babel}
\usepackage[latin1]{inputenc}
\usepackage[T1]{fontenc}

% change the alerted colour to LimeGreen
\definecolor{LimeGreen}{RGB}{50,205,50}
\setbeamercolor{structure}{fg=LimeGreen}
\author[FOSSEE]{}
\institute[IIT Bombay]{}
\date[]{}
% \setbeamercovered{transparent}

% theme split
\usepackage{verbatim}
\newenvironment{colorverbatim}[1][]%
{%
\color{blue}
\verbatim
}%
{%
\endverbatim
}%

\usepackage{mathpazo,courier,euler}
\usepackage{listings}
\lstset{language=sh,
    basicstyle=\ttfamily\bfseries,
  showstringspaces=false,
  keywordstyle=\color{black}\bfseries}

% logo
\logo{\includegraphics[height=1.30 cm]{../images/3t-logo.pdf}}
\logo{\includegraphics[height=1.30 cm]{../images/fossee-logo.pdf}

\hspace{7.5cm}
\includegraphics[scale=0.99]{../images/fossee-logo.pdf}\\
\hspace{281pt}
\includegraphics[scale=0.80]{../images/3t-logo.pdf}}
%%%%%%%%%%%%%%%%%%%%%%%%%%%%%%%%%%%%%%%%%%%%%%%%%%%%%%%%%%%%%%%%%%%%%%
% DOCUMENT STARTS
\begin{document}

\sffamily \bfseries
\title
[Redirection and Piping]
{Redirection and Piping}
\author
[FOSSEE]
{\small Talk to a Teacher\\{\color{blue}\url{http://spoken-tutorial.org}}\\\vspace{0.25cm}National Mission on Education
 through ICT\\{\color{blue}\url{ http://sakshat.ac.in}} \\ [1.65cm]
   Contributed by FOSSEE Team \\IIT Bombay  \\[0.3cm]
}

% slide 1
\begin{frame}
   \titlepage
\end{frame}

\begin{frame}
\frametitle{Objectives}
\label{sec-2}

At the end of this tutorial, you will be able to,
\begin{itemize}
\item Understand what is Redirection
\item Learn the concept of Piping
\end{itemize}
\end{frame}

\begin{frame}
\frametitle{Pre-requisites}
\label{sec-3}

Spoken tutorial on,
\begin{itemize}
\item Getting started with Linux
\item Basic File Handling
\item Advanced File handling
\end{itemize}
\end{frame}

\begin{frame}[fragile]
  \frametitle{Redirection} 

  \begin{itemize}
  \item The standard output (stdout) stream goes to the display
  \item May not be always, what we need
  \item \texttt{>} states that output is redirected to the specified location 
  \item It is followed by location to redirect,
  \end{itemize}
  \begin{lstlisting}
    $ command > file1
  \end{lstlisting} % $
\end{frame}

\begin{frame}[fragile]
  \frametitle{Redirection..} 
  \begin{itemize}
  \item Similarly, the standard input (stdin) can be redirected as
  \end{itemize}
  \hspace{29pt}\texttt{\$ command < file1}
  \begin{itemize}
  \item input and output redirection could be combined
  \end{itemize}
  \hspace{29pt}\texttt{\$ command < infile > outfile}
\end{frame}

\begin{frame}
\frametitle{stderr}
  \begin{itemize}
  \item Standard error (stderr) is the third standard stream
  \item All error messages come through this stream
  \item \texttt{stderr} can also be redirected
  \end{itemize}
\end{frame}

\begin{frame}[fragile]
\frametitle{Piping}
\begin{lstlisting}
$ cut -d " " -f 2- marks1.txt 
  | paste -d " " students.txt -
  \end{lstlisting} % $
  \begin{itemize}
  \item \texttt{-} at the end asks \texttt{paste} to read from
    \texttt{stdin} instead of FILE 
  \item \texttt{cut} command here is a normal command
  \end{itemize}
\end{frame}

\begin{frame}[fragile]
\frametitle{Piping..}
\begin{itemize}
  \item the \texttt{|} seems to be joining the two commands
  \item Redirects output of first command to \texttt{stdin}, which
    becomes input to the second command
  \item This is called piping; \texttt{|} is called a pipe
  \end{itemize}
\end{frame}

\begin{frame}[fragile]
  \frametitle{Piping..}
  \begin{itemize}
  \item Roughly same as -- two redirects and a temporary file
  \end{itemize}
  \begin{lstlisting}
    $ command1 > tempfile
    $ command2 < tempfile
    $ rm tempfile
\end{lstlisting} % $
\begin{itemize}
\item Any number of commands can be piped together
\end{itemize}
\end{frame}


\begin{frame}
\frametitle{Summary}
\label{sec-8}

  In this tutorial, we have learnt to,


\begin{itemize}
\item Use the ``cut'' and ``paste'' commands in redirection
\item Apply the concept of Piping
\end{itemize}
\end{frame}
\begin{frame}[fragile]
\frametitle{Evaluation}
\label{sec-9}


\begin{enumerate}
\item How to redirect content from file to device ?
\vspace{12pt}
\item How to view last field (30), in a file located at \verb~/home/test.txt~ 
whose first line is "data:myscripts:20:30"
\vspace{5pt}
\begin{itemize}
\item cut -d : -f 4 /home/test.txt
\item cut -f 3 /home/test.txt
\item cut -d : -f 3 /home/test.txt
\end{itemize}
\end{enumerate}
\end{frame}
\begin{frame}
\frametitle{Solutions}
\label{sec-10}


\begin{enumerate}
\item \$ cat filename > device\\
For eg:\\
\hspace{8pt} \verb~cat sound.wav > /dev/audio~    
\vspace{22pt}
\item \$ cut -d : -f 4 /home/test.txt
\end{enumerate}
\end{frame}

\begin{frame}
\frametitle{SDES \& FOSSEE}
\begin{center}
\begin{itemize}
\item \small{SDES}\\
\small{\color{LimeGreen}Software Development techniques for Engineers and Scientists} \\
\scriptsize An initiative by FOSSEE. \\
\vspace{3pt}
\scriptsize For more information on SDES, please visit {\color{blue}\url{http://fossee.in/sdes}}\\
\vspace{10pt}
\item \small{FOSSEE}\\
\small {\color{LimeGreen}Free and Open-source Software for \\Science and Engineering Education} \\
\scriptsize Based at IIT Bombay, Funded by MHRD.\\
\vspace{3pt}
\scriptsize Part of National Mission on Education through ICT \\(NME-ICT) \\
\end{itemize}
\end{center}
\end{frame}

\begin{frame}
\frametitle{About the Spoken Tutorial Project}
\begin{itemize}
\item Watch the video available at {\color{blue}\url{http://spoken-tutorial.org /What\_is\_a\_Spoken\_Tutorial}} 
\item It summarises the Spoken Tutorial project 
\item If you do not have good bandwidth, you can download and watch it
\end{itemize}
\end{frame}

\begin{frame}
\frametitle{Spoken Tutorial Workshops}The Spoken Tutorial Project Team 
\begin{itemize}
\item Conducts workshops using spoken tutorials 
\item Gives certificates to those who pass an online test 
\item For more details, please write to \\ \hspace {0.5cm}{\color{blue}contact@spoken-tutorial.org}
\end{itemize}
\end{frame}

\begin{frame}
\frametitle{Acknowledgements}
\begin{itemize}
\item Spoken Tutorial Project is a part of the Talk to a Teacher  project 
\item It is supported by the National Mission on Education through  ICT, MHRD, Government of India 
\item More information on this Mission is available at: \\{\color{blue}\url{http://spoken-tutorial.org/NMEICT-Intro}}
\end{itemize}
\end{frame}

\begin{frame}

  \begin{block}{}
  \begin{center}
  {\Large THANK YOU!} 
  \end{center}
  \end{block}
\begin{block}{}
  \begin{center}
    For more Information, visit our website\\
    {\color{blue}\url{http://fossee.in/}}
  \end{center}  
  \end{block}
\end{frame}


\end{document}


